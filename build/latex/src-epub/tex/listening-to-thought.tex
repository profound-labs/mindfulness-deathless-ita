
Praticando l'apertura mentale - ossia ``lasciando andare'' - portiamo
l'attenzione sul semplice fatto del guardare, dell'essere il testimone
silenzioso che è consapevole di quello che viene e va. In questo tipo di
meditazione, che si definisce \textit{vipassanā}, osserviamo i fenomeni fisici e
mentali alla luce delle tre caratteristiche: \textit{anicca} (cambiamento),
\textit{dukkha} (carattere insoddisfacente), \textit{anattā} (impersonalità). così facendo
liberiamo la mente dalla tendenza a reprimere ciecamente; dunque se ci
troviamo ossessionati da pensieri banali o paure, sentimenti di
preoccupazione o di rabbia, non è necessario analizzarli. Non dobbiamo
capire perché li abbiamo, ma solo farli emergere pienamente alla
coscienza.

Se siete molto spaventati, siate spaventati consciamente. Non
ritraetevi, ma notate la tendenza a volervi sbarazzare della paura. Fate
emergere l'oggetto della vostra paura, pensateci deliberatamente, e
ascoltate i vostri pensieri. Non si tratta di analizzarli, ma di portare
agli estremi la paura, al punto in cui diventa così assurda da poterne
ridere. Ascoltate il desiderio, la furia del ``voglio questo, voglio
quello, devo averlo, cosa farò se non riesco ad averlo, lo voglio
assolutamente...''. A volte nella mente c'è solo un grido inarticolato:
``voglio!'' - ed è possibile ascoltarlo.

Ho letto qualcosa sulle tecniche del confronto terapeutico, sapete,
quelle situazioni in cui ci si gridano in faccia a vicenda tutti i
sentimenti repressi; l'effetto vorrebbe essere catartico, ma manca la
saggia riflessione. Manca la capacità di ascoltare quel grido come una
condizione della mente, invece di ``lasciarsi andare'' a dire tutto quello
che passa per la testa. Manca l'equilibrio mentale, la disponibilità a
tollerare anche i pensieri più orribili. Così facendo, non li
consideriamo come problemi personali, ma piuttosto portiamo all'assurdo
la rabbia e la paura, al punto in cui vengono viste come una naturale
catena di pensieri. Ci mettiamo a pensare deliberatamente a tutto quello
che abbiamo paura di pensare, non ciecamente, ma osservando e ascoltando
quei pensieri in quanto condizioni della mente, piuttosto che come
difetti o problemi personali.

Sicché, con questa pratica cominciamo a lasciar andare. Non c'è bisogno
di andare a cercare qualcosa in particolare; ma se vi sentite
infastiditi da contenuti che tendono a riemergere ossessivamente e che
cercate di allontanare, fateli venire a galla ancora di più. Pensateci
deliberatamente e restate in ascolto, come ascoltereste qualcuno che
parla dall'altro lato del cortile, una vecchia pescivendola pettegola:
``Abbiamo fatto questo, e poi quest'altro, e abbiamo fatto questo e poi
quest'altro...'' quella vecchietta che non la finisce più di
chiacchierare! Ora esercitatevi ad ascoltarla come una voce e basta,
invece di giudicarla, invece di dire: ``No no, spero proprio di non
essere io, che non sia la mia vera natura'', o cercare di tapparle la
bocca: ``Ma quando la finisci vecchia strega!''. Facciamo tutti quanti un
po' così, anche io ho questa tendenza. Ma è solo una condizione della
natura, vi pare? Non una persona. Sicché, questa abitudine fastidiosa
dentro di noi: ``Mi ammazzo di lavoro e mai nessuno che mi dica grazie'' –
è una condizione, non una persona. A volte, quando esiste una condizione
di malumore, nessuno fa le cose come si deve, e anche se prova a farle
non va bene lo stesso! Anche questa è una condizione della mente, non
una persona. Il malumore, l'irritabilità della mente, viene riconosciuto
come una condizione: \textit{anicca}, è impermanente; \textit{dukkha}, è insoddisfacente;
\textit{anattā}, è impersonale. C'è la paura di quello che penseranno gli altri
se arrivate tardi: avete dormito troppo, entrate nella stanza e
cominciate a preoccuparvi di quello che pensano gli altri del vostro
ritardo - ``Penseranno che sono pigro''. Preoccuparsi del giudizio degli
altri è una condizione della mente. Oppure siamo sempre puntuali e
qualcun altro arriva in ritardo: ``Sempre in ritardo, mai una volta che
arrivi puntuale''. Anche questa è una condizione della mente.

Allora faccio emergere il tutto alla piena coscienza, queste cose banali
che si possono benissimo trascurare perché tanto sono banali, e non
abbiamo voglia di avere a che fare con le banalità della vita; ma quando
non vogliamo averci a che fare tutto questo viene represso, e diventa un
problema. Cominciamo a sentirci in ansia, ostili a noi stessi o agli
altri, o subentra la depressione; tutti effetti del nostro rifiuto di
lasciare che le condizioni, banali o orribili che siano, emergano alla
coscienza.

Poi c'è lo stato mentale del dubbio, la perenne incertezza sul da farsi:
c'è timore e dubbio, insicurezza ed esitazione. Fate emergere
deliberatamente quello stato di perenne incertezza, solo per imparare a
rilassarvi con lo stato in cui si trova la mente quando non è attaccata
a nulla in particolare. ``Che devo fare? Restare o andarmene? Dovrei fare
questo oppure quest'altro, devo praticare \textit{ānāpānasati} oppure la
\textit{vipassanā}?''. Osservatelo. Ponetevi domande senza risposta, tipo ``Chi
sono?''. Notate lo spazio vuoto che precede il pensiero ``chi'' - restate
vigili, chiudete gli occhi e un attimo prima di pensare ``chi'' osservate:
la mente è vuota, vero? Poi: ``Chi-sono-io?'', seguito dallo spazio dopo
il punto interrogativo. Quel pensiero nasce dal vuoto e torna al vuoto,
no? Quando siete presi dal pensiero abituale non potete scorgere
l'origine del pensiero, vero? Non potete, potete cogliere il pensiero
solo dopo esservi accorti di stare pensando; quindi cominciate a pensare
deliberatamente, e cogliete il principio di un pensiero, prima di
cominciare a pensarlo effettivamente. Prendete un pensiero deliberato,
tipo: ``Chi è il \textit{Buddha}?''. Pensatelo deliberatamente, in modo da
percepire l'inizio, il formarsi del pensiero e poi la fine, e lo spazio
che lo circonda. Si tratta di osservare pensieri e concetti in
prospettiva, invece di limitarsi a reagire alla loro presenza.

Mettiamo che siate arrabbiati con qualcuno. ``Ecco che ha detto, ha detto
questo e quest'altro, e ha fatto così e colà e non ha fatto bene, ha
sbagliato tutto, è un vero egoista... ricordo ancora quello che ha fatto
al tal dei tali, e poi...''. Un pensiero tira l'altro, vero? E vi
ritrovate coinvolti in questa catena di pensieri motivata
dall'avversione. Perciò, invece di farvi coinvolgere in tutta una serie
di associazioni e concetti, pensate deliberatamente: ``E' la persona più
egoista che abbia mai conosciuto''. Poi fine, il vuoto. ``E' un bastardo,
un disgraziato, ha fatto questo e quest'altro''; a quel punto è veramente
comico, vi pare? Appena arrivato al Wat Pah Pong\footnote{Wat Pah Pong: il monastero thailandese dove insegnava
Ajahn Chah.} sperimentavo
fortissimi sentimenti di rabbia e di avversione. Mi sentivo
terribilmente frustrato, a volte perché non capivo cosa succedeva
intorno a me e non volevo uniformarmi tanto quanto mi veniva richiesto.
Ero letteralmente furente. Ajahn Chah tirava avanti imperterrito -
discorsi di due ore filate in laotiano - e le ginocchia mi facevano male
da morire. Sicché pensavo cose come: ``perché non la finisci di parlare?
Pensavo che il \textit{Dhamma} fosse semplice, perché deve metterci due ore per
spiegare un concetto?''. Ero ipercritico nei confronti di tutti, ma poi
cominciai a contemplare questo e ad ascoltarmi, la mia rabbia, le mie
critiche, le mie cattiverie, il mio risentimento: ``questo non mi va,
quell'altro non mi va, non capisco perché devo sedermi qui, non voglio
occuparmi di sciocchezze del genere, non so proprio''... e così via
all'infinito. Mi ripetevo: ``Ti pare simpatico uno che dice cose del
genere? E' questo che hai deciso di essere, questa cosa che sta sempre a
lamentarsi a criticare e trovare difetti, e così che vuoi essere?'', ``No
- mi rispondevo - non voglio essere così''.

Ma prima ho dovuto far venire a galla tutto per vederlo davvero,
piuttosto che crederci in teoria. Sentivo di aver ragione da vendere, e
quando uno sente di avere ragione, e si indigna, e pensa che gli altri
abbiano torto, è portato a dare credito a pensieri come: ``In fin dei
conti non vedo che motivo ci sia ... il \textit{Buddha} ha detto... il \textit{Buddha},
lui, non lo avrebbe mai permesso, io lo conosco il Buddhismo!''. Fatelo
emergere in forma cosciente, dove potete vederlo, portarlo all'assurdo,
così potrete guardarlo in prospettiva e alla fine vi sembrerà comico.
Capite che è tutta una commedia! Ci prendiamo terribilmente sul serio:
``Sono una persona veramente importante, la mia vita e così tremendamente
importante che devo prenderla estremamente sul serio sempre e comunque.
I miei problemi sono veramente importanti, terribilmente importanti.
Devo dedicare un sacco di tempo ai miei problemi perché sono davvero
importanti''. In un modo o nell'altro ci riteniamo importantissimi,
perciò pensatelo deliberatamente: ``Sono una Persona Molto Importante, i
miei problemi sono molto importanti e seri''. Quando fate così, il tutto
prende un aspetto comico; appare sciocco, perché vi rendete conto che in
definitiva non siete terribilmente importanti, nessuno di noi lo è. E i
problemi che ci creiamo nella vita sono banalità. C'è gente che si
rovina l'esistenza generando problemi a non finire, e prendendo tutto
estremamente sul serio.

Se vi ritenete persone importanti e serie, le cose banali o futili vi
sembreranno inaccettabili. Se aspirate a essere buoni, a essere santi,
sarete portati a escludere dalla coscienza gli stati mentali negativi.
Se desiderate essere persone amorevoli e generose, ogni forma di
meschinità, di invidia o di avarizia dovrete reprimerla o estrometterla
dalla vostra mente. Sicché, se c'è qualcosa che temete sopra ogni altra
di poter diventare davvero nella vostra vita, pensatela, guardatela.
Confessatelo apertamente: ``Voglio essere un tiranno; voglio essere uno
spacciatore di eroina; voglio essere un mafioso''; sia quel che sia. Non
ci interessa più il contenuto specifico, ma la semplice caratteristica
di essere una condizione impermanente, insoddisfacente, perché non ha
nulla che potrà darvi una reale soddisfazione. Viene e va, ed è non-sé.
