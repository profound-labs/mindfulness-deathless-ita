
Nel corso della prossima ora praticheremo la meditazione camminata,
prendendo come oggetto di concentrazione l'azione del camminare,
portando l'attenzione al movimento dei piedi e alla pressione dei piedi
sul terreno. Si può anche associare alla camminata il mantra \textit{Buddho}:
``Bud'' sul piede destro, ``dho'' sul sinistro, per tutta la lunghezza del
sentiero del \textit{joṅgrom}. Vedete se vi riesce di rimanere in contatto con la
sensazione, pienamente svegli alla sensazione di camminare, dal
principio alla fine del tratto del \textit{joṅgrom}. Camminate a velocità
normale, potrete poi rallentare o aumentare l'andatura a seconda dei
casi. Tenete un'andatura normale, perché la nostra meditazione si
interessa alle cose ordinarie, più che a quelle speciali. Usiamo il
respiro normale, non uno speciale ``esercizio di respirazione'', la
posizione seduta piuttosto che tenerci in equilibrio sulla testa,
un'andatura normale, invece di correre, saltellare sul posto o camminare
a passo deliberatamente rallentato: semplicemente un'andatura rilassata.
La nostra pratica ha per oggetto l'ordinario, perché lo diamo per
scontato. Ora però portiamo la nostra attenzione a tutte le cose che
abbiamo dato per scontate e non abbiamo mai notato, come la nostra mente
e il nostro corpo. Perfino i medici esperti di fisiologia e anatomia non
sono realmente in contatto con il proprio corpo. Ci dormono insieme, ci
sono nati insieme, invecchiano, devono conviverci, nutrirlo,
esercitarlo, eppure vi parleranno del fegato come di quella cosa che sta
sugli atlanti di anatomia. E' più facile guardare un fegato disegnato
che essere consapevoli del proprio, non è vero? Sicché noi guardiamo il
mondo come se in un certo senso non ne facessimo parte, e ciò che è più
ordinario, più comune, ci sfugge, interessati come siamo allo
straordinario.

La televisione è lo straordinario. Alla televisione fanno ogni sorta di
cose fantastiche avventurose e romantiche. E' un oggetto miracoloso, sul
quale quindi è facile concentrarsi. Davanti alla TV è facile restare
ipnotizzati. Anche quando il corpo diventa straordinario - quando è
molto malato o molto dolorante o prova sensazioni esaltanti o
meravigliose - allora sì che lo notiamo! Ma la semplice pressione del
piede destro sul terreno, il semplice movimento del respiro, la semplice
sensazione del corpo seduto sulla sedia quando non c'è nessuna
sensazione estrema - è a questo tipo di cose che ci risvegliamo.
Portiamo la nostra attenzione alle cose così come sono in un'esistenza
ordinaria.

Quando la vita prende forme estreme, o straordinarie, ce la sappiamo
cavare benissimo. Spesso i pacifisti e gli obiettori di coscienza si
sentono rivolgere la fatidica domanda: ``Voi rifiutate la violenza; ma
cosa fareste se un maniaco aggredisse vostra madre?''. Ecco un dilemma
che non credo si sia posto spesso alla maggior parte di noi! Non è il
tipo di evento che capita di norma nella vita quotidiana. Ma se una
situazione così grave si dovesse verificare, sono sicuro che reagiremmo
nella maniera più opportuna. Anche il più tonto sa essere presente a se
stesso in circostanze estreme. Ma nella vita normale, quando non succede
nulla di grave, come ora che siamo semplicemente seduti qui, possiamo
permetterci di essere completamente tonti, no? Nel \textit{Pāṭimokkha}\footnote{\textit{Pāṭimokkha}: il codice di 227 regole e prescrizioni che
disciplina la condotta dei monaci buddhisti della tradizione \textit{Theravāda}.}
si dice che il monaco non deve picchiare nessuno. Ecco allora che
comincio a preoccuparmi di cosa farei se un maniaco aggredisse mia
madre. Ho creato un grosso problema morale in una situazione ordinaria,
mentre sono seduto qui e mia madre non è neppure presente. In tutti
questi anni nessun maniaco ha mai attentato all'incolumità di mia madre
(diversamente dagli automobilisti californiani!). Le grosse questioni
morali si risolvono al momento e nel luogo opportuni, a condizione che,
adesso, siamo consapevoli di questo momento e di questo luogo.

Dunque stiamo portando l'attenzione sull'ordinarietà della nostra
condizione umana; il corpo che respira, camminare da un punto all'altro
sul sentiero del \textit{joṅgrom}, le sensazioni di piacere e dolore. Nel corso
del ritiro prendiamo in esame assolutamente tutto, lo osserviamo e lo
conosciamo per quello che è. Questa è la nostra pratica di \textit{vipassanā}:
conoscere le cose così come sono, non in base a una teoria o un assunto
creati da noi.

