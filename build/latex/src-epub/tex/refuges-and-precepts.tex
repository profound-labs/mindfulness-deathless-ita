

``Prendere rifugio'' e osservare i ``cinque precetti'' definisce una persona
come un praticante buddhista.

La presa di rifugio ci offre una prospettiva di fondo sulla vita,
riferendo la nostra condotta e la nostra comprensione alle qualità del
\textit{Buddha} (saggezza), \textit{Dhamma} (verità) e \textit{Sangha} (virtù). I precetti inoltre
incoraggiano la riflessione e l'assunzione di responsabilità nelle
proprie azioni.

Tradizionalmente, i Rifugi e i Precetti si possono richiedere
formalmente a un \textit{bhikkhu} (monaco), secondo la formula che seguente.

Il laico si inchina tre volte, e con le mani giunte in
\textit{añjali}\footnote{\textit{Añjali}: gesto di saluto, omaggio e rispetto, in cui si
tengono le mani giunte in posizione verticale vicino al petto.} recita:
Ahaµ bhante tisara¼ena saha pañca sølæni yæcæmi

Dutiyampi ahaµ bhante tisara¼ena saha pañca sølæni yæcæmi

Tatiyampi ahaµ bhante tisara¼ena saha pañca sølæni yæcæmi

Io, venerabile signore, chiedo i Tre Rifugi e i Cinque Precetti.

Per la seconda volta, venerabile signore, chiedo i Tre Rifugi e i Cinque
Precetti.

Per la terza volta, venerabile signore, chiedo i Tre Rifugi e i Cinque
Precetti.

Poi il monaco recita il verso seguente tre volte, e il laico ripete tre
volte dopo di lui:

Namo tassa bhagavato arahato sammæsambuddhassa

Namo tassa bhagavato arahato sammæsambuddhassa

Namo tassa bhagavato arahato sammæsambuddhassa

Rendo omaggio al Beato, al Nobile, perfettamente risvegliato.

Rendo omaggio al Beato, al Nobile, perfettamente risvegliato.

Rendo omaggio al Beato, al Nobile, perfettamente risvegliato.

Il monaco recita quanto segue, un verso alla volta, e il laico ripete
dopo di lui:

\textit{Buddha}µ sara¼aµ gacchæmi Prendo rifugio nel \textit{Buddha}

\textit{Dhamma}µ sara¼aµ gacchæmi Prendo rifugio nel \textit{Dhamma}

Sa¼ghaµ sara¼aµ gacchæmi Prendo rifugio nel \textit{Sangha}

Dutiyampi buddhaµ sara¼aµ gacchæmi Per la seconda volta, prendo rifugio
nel \textit{Buddha}

Dutiyampi dhammaµ sara¼aµ gacchæmi Per la seconda volta, prendo rifugio
nel \textit{Dhamma}

Dutiyampi sa¼ghaµ sara¼aµ gacchæmi Per la seconda volta, prendo rifugio
nel \textit{Sangha}

Tatiyampi buddhaµ sara¼aµ gacchæmi Per la terza volta, prendo rifugio
nel \textit{Buddha}

Tatiyampi dhammaµ sara¼aµ gacchæmi Per la terza volta, prendo rifugio
nel \textit{Dhamma}

Tatiyampi sa¼ghaµ sara¼aµ gacchæmi Per la terza volta, prendo rifugio
nel \textit{Sangha}

Quindi il monaco dice:

Tisara¼a-gamanaµ ni¥¥hitaµ La presa dei Tre Rifugi è completa.

Il laico risponde:

âma bhante Sì, venerabile signore.

Poi il monaco recita i precetti uno dopo l'altro, e il laico ripete dopo
di lui:

\1. Pæ¼ætipætæ verama¼ø sikkhæpadaµ samædiyæmi

Prendo il precetto di astenermi dal distruggere intenzionalmente
qualunque creatura vivente.

\2. Adinnædænæ verama¼ø sikkhæpadaµ samædiyæmi

Prendo il precetto di astenermi dal prendere ciò che non è stato dato
liberamente.

\3. Kæmesu micchæcæræ verama¼ø sikkhæpadaµ samædiyæmi

Prendo il precetto di astenermi da una cattiva condotta sessuale.

\4. Musævædæ verama¼ø sikkhæpadaµ samædiyæmi

Prendo il precetto di astenermi da un linguaggio scorretto.

\5. Suræmeraya-majja-pamæda¥¥hænæ verama¼ø sikkhæpadaµ samædiyæmi

Prendo il precetto di astenermi da bevande intossicanti o da droghe che
alterano la mente.

In conclusione il monaco dice:

Imæni pañca sikkhæpadæni Questi cinque precetti

Sølena sugatiµ yanti Hanno la moralità come veicolo di felicità

Sølena bhogasampadæ Hanno la moralità come veicolo di benessere

Sølena nibbutiµ yanti Hanno la moralità come veicolo di liberazione

Tasmæ sølaµ visodhaye Sia dunque resa pura la condotta morale.

Il laico risponde:

Sædhu, Sædhu, Sædhu (E' bene, è bene, è bene!) e poi si inchina tre
volte.
