
Il \textit{joṅgrom}\footnote{\textit{Joṅgrom}: vedi nota introduttiva a pag \pageref{jongrom}.} si associa ad una pratica in cui si cammina in
modo raccolto, restando in contatto con il movimento dei piedi. Si porta
l'attenzione sul corpo che cammina dal principio del sentiero fino alla
fine, all'atto di voltarsi, al corpo fermo in piedi. Poi c'è
l'intenzione di riprendere a camminare, quindi il camminare. Fate
attenzione alla metà del percorso e alla fine, al fermarsi, al voltarsi
e allo stare fermi in piedi: sono i punti su cui la mente può
raccogliersi quando comincia a disperdersi. Si può anche progettare una
rivoluzione, o che so io, durante la camminata sul \textit{joṅgrom}, se non si
sta attenti! Chissà quante rivoluzioni sono state tramate durante la
meditazione camminata... Perciò, invece di dedicarci a cose del genere,
usiamo questo tempo per concentrarci su ciò che accade effettivamente. E
non sono sensazioni straordinarie, sono così normali che neppure ci
facciamo caso. Notate che bisogna fare uno sforzo per essere veramente
consapevoli di questo tipo di cose.

Ora, quando la mente si distrae e vi ritrovate in India nel bel mezzo
del sentiero del \textit{joṅgrom}, accorgetevene: ``Oh!'', in quel momento vi
risvegliate. Siete svegli, quindi riportate la mente su quanto sta
realmente accadendo mentre il corpo cammina da qui a lì. E' un
allenamento alla pazienza, perché la mente tende ad andarsene per conto
suo. Se in passato la meditazione camminata vi ha dato momenti di
felicità e pensate: ``Nell'ultimo ritiro ho praticato sul \textit{joṅgrom} e ho
proprio sentito il corpo che camminava. Sentivo che non c'era alcun ``io''
ed e stato bellissimo, oh, se non riesco a farlo ancora...'', notate il
desiderio di raggiungere qualcosa in base al ricordo di un passato
felice. Notate che è un condizionamento; e, di fatto, un ostacolo.
Lasciate andare, non ha importanza se ne viene fuori un istante di
beatitudine o no. Solo un passo e poi un altro passo - la pratica è
tutta qui, nel lasciar andare, nel contentarsi di pochissimo invece di
sforzarsi di ritrovare uno stato di beatitudine sperimentato in passato
grazie a questa meditazione. Più vi sforzate, più l'infelicità aumenta
perché assecondate il desiderio di avere una bella esperienza fondata su
un ricordo. Siate contenti di quello che c'è, qualunque cosa sia. Fate
pace con le cose così come sono adesso, invece di arrabattarvi per
ottenere a tutti costi un certo stato desiderato.

Un passo alla volta - notate com'è serena la meditazione camminata
quando non avete altro da fare che essere con un passo. Ma se invece
fate la camminata pensando di dover raggiungere il \textit{samādhi} e poi
l'attenzione divaga, che succede? ``Io la meditazione camminata non la
sopporto, non mi dà nessuna pace, ho tentato di avere la sensazione di
camminare senza nessuno che cammina ma invece pensavo a tutt'altro'';
questo perché ancora non capite di che si tratta, la mente proietta un
ideale, cerca di ottenere qualcosa invece di limitarsi ad essere. Quando
camminate, non dovete far altro che camminare. Un passo, poi un altro
passo - semplice... ma non facile, vero? La mente è trascinata lontano,
preoccupata di cosa dovreste fare, di cosa non va e del perché non ci
riuscite.

Ma al monastero quello che facciamo è alzarci al mattino, recitare i
canti, meditare, sederci, fare le pulizie, prepararci per il pranzo,
sederci, stare in piedi, camminare, lavorare; qualunque cosa sia,
prendetela come viene, una cosa alla volta. Sicché, stare con le cose
così come sono è non-attaccamento, che porta pace e serenità. La vita
cambia e noi la guardiamo cambiare, possiamo adattarci alla mutevolezza
del mondo sensoriale, come che sia. Piacevole o spiacevole, abbiamo
sempre la possibilità di tollerare e affrontare la vita, accada quel che
accada. Se realizziamo la verità, realizziamo la pace interiore.
