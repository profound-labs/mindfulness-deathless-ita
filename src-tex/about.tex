
{\centering\par
\Large\scshape\chapTitleFont\thetitle
\par}
\vspace*{2\baselineskip}

{%
\setlength{\parskip}{1.5em}
\setlength{\parindent}{0pt}

The material in this book has been edited from talks given at Chithurst Forest Monastery in January 1984, with the exception of three sections. These are `\textit{Effort and Relaxation},' `\textit{Kindness},' `\textit{The Hindrances and Their Cessation},' which were taken from talks given at Wat Pah Nanachat International Forest Monastery, Ubon, N.E. Thailand in December 1982.

The photographs are of slabs from the ruins of the ancient \textit{stūpa} at Amaravati in Andhra Pradesh, India; they are reproduced by kind permission of the Trustees of the British Museum, London.

Page \pageref{image-stupa}: the \textit{stūpa}, a monumental reliquary, contains the relics of a saint. As the object of pilgrimages, it symbolizes the universal quest for spiritual truth.

Page \pageref{image-feet}: the iconographical footprints of the Buddha. They represent the path that a teacher has taken, to be followed by his disciples.

Page \pageref{image-lotus-scene}: the lotus of wisdom in a scene of everyday human activity.

\clearpage
\thispagestyle{empty}

{\scshape \theauthor} è nato negli USA, a Seattle nel 1934. Ha
preso l'ordinazione come bhikkhu (monaco della tradizione Theravāda) in
Thailandia nel 1966 e ha trascorso dieci anni nel nord-est del paese con
il Venerabile Ajahn Chah, insegnante della tradizione spirituale dei
“Maestri della Foresta”. Invitato nel 1976 in Inghilterra da
un'associazione laica (English Sangha Trust) ha successivamente fondato
i monasteri di Chithurst e di Amaravati in Inghilterra, inoltre ha
incoraggiato e sostenuto l'apertura di altri monasteri, fra cui il
Santacittarama in Italia, contribuendo a diffondere la tradizione
Theravāda in diversi paesi occidentali. Nel 2010, dopo 33 anni di
servizio nelle comunità occidentali, ha deciso di lasciare i suoi
incarichi pubblici e ritornare in Thailandia per continuare la sua vita
monastica in forma più ritirata.

{\scshape Ajahn Sucitto} is the abbot of Cittaviveka, Chithurst Buddhist Monastery since 1992. He contributed the Introduction to this booklet in 1986. He was born in London, and was ordained in Thailand in March 1976. He moved to Britain in 1978 and took up training under Ajahn Sumedho at the Hampstead Buddhist Vihara. In 1979 he was one of the small group of monks, led by Ajahn Sumedho, who established Cittaviveka, Chithurst Buddhist Monastery, in West Sussex.


}

