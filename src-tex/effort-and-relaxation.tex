% Effort and relaxation

Sforzo significa semplicemente fare quel che c'è da fare. Varia a
seconda delle caratteristiche e delle abitudini individuali. Alcune
persone hanno molta energia, così tanta che sono sempre in movimento, in
cerca di attività da svolgere. Sono quelli che si danno da fare
continuamente, sempre proiettati all'esterno. In meditazione non ci
proponiamo di sperimentare nulla come un mezzo di fuga, ma piuttosto
coltiviamo uno sforzo di natura interiore. Osserviamo la mente, e ci
concentriamo sul soggetto.

Se lo sforzo è eccessivo si finisce col diventare irrequieti, se
viceversa non ci si sforza abbastanza la mente si annebbia e il corpo
tende ad accasciarsi. Il corpo è un buon parametro per misurare lo
sforzo: lo si mantiene eretto, si può infondere energia nel corpo;
tenetelo allineato, tirate su il petto, tenete diritta la spina dorsale.
Ciò richiede una grossa dose di forza di volontà, quindi è una buona
iniziativa mantenere presente il corpo per valutare lo sforzo. Se siete
pigri, sceglierete la postura più facile, e la forza di gravità vi
tirerà giù. Quando fa freddo bisognerà stirare energicamente la spina
dorsale per dare al corpo pienezza e slancio, invece di rannicchiarsi
sotto le coperte.

Nell'\textit{ānāpānasati}, la consapevolezza del respiro, ci si concentra sul
ritmo. Io lo trovo particolarmente utile per imparare a rallentare
invece di fare tutto rapidamente – alla velocità del pensiero; ci si
concentra su un ritmo che è molto più lento di quello del pensiero.
L'\textit{ānāpānasati} vi chiede di rallentare, ha un ritmo dolce. Quindi
smettiamo di pensare: ci accontentiamo di un'inspirazione,
un'espirazione, prendendoci tutto il tempo del mondo solo per essere con
un'inspirazione, dal principio alla metà, alla fine.

Se praticando \textit{ānāpānasati} vi sforzate di raggiungere il \textit{samādhi}, vi
siete già posti un obiettivo, lo state facendo per ottenere qualcosa,
quindi la pratica diventa un'esperienza molto frustrante che tenderà a
suscitarvi rabbia. Riuscite a stare semplicemente con un'inspirazione? A
contentarvi di una semplice espirazione? Per contentarvi di quel
semplice piccolo lasso di tempo dovete rallentare, vero?

Quando mirate a ottenere il \textit{jhāna}\footnote{\textit{Jhāna}: in genere tradotto come “assorbimento”, indica
uno stato di concentrazione mentale profondo.} e vi sforzate al massimo
per ottenerlo, non state rallentando, state cercando di ricavarne
qualcosa, di conquistare e ottenere invece di contentarvi umilmente di
un respiro. Il successo dell'\textit{ānāpānasati} è tutto qui: essere consapevoli
per la durata di un'inspirazione, per la durata di un'espirazione.
Ponete l'attenzione al principio e alla fine, oppure al principio, alla
metà e alla fine. Avrete così dei punti di riferimento precisi per la
contemplazione, di modo che se nel corso della pratica tendete a
distrarvi parecchio potete applicare una particolare attenzione
all'osservazione del principio, della metà e della fine. Diversamente,
la mente tenderà a divagare.

Tutti i nostri sforzi tendono solo a questo; tutto il resto è
momentaneamente annullato o messo da parte. Riflettete sulla differenza
fra inspirazione ed espirazione, esaminatela. Quale preferite? A volte
sembrerà che il respiro scompaia, diventando molto sottile. Sembra che
il corpo respiri da solo e si ha la strana sensazione di non respirare
più. Fa un po' paura.

Ma è solo un esercizio; centratevi sul respiro, senza cercare di
controllarlo. A volte, concentrandovi sulle narici, vi sembrerà che
tutto il corpo respiri. Il corpo continua a respirare, per conto suo.

A volte prendiamo tutto troppo sul serio, manchiamo completamente di
gioia e felicità, di senso dell'umorismo; ci reprimiamo e basta. Perciò,
rasserenatevi, rilassatevi e state tranquilli, prendendovi tutto il
tempo del mondo, senza l'assillo di dover raggiungere qualcosa di
importante; non è niente di speciale, niente da ottenere, niente da
guadagnare. E' una esperienza senza pretese; anche una sola inspirazione
consapevole in tutta la mattinata è sempre meglio di quello che fa la
maggior parte della gente, sicuramente meglio che essere distratti per
tutto il giorno. Se avete un carattere ipercritico, cercate di essere
più gentili e più tolleranti nei vostri confronti. Rilassatevi e non
prendete la meditazione come un compito gravoso. Consideratela
un'occasione per essere in pace e a vostro agio con il momento presente.
Rilassate il corpo e siate sereni.

Non siete in guerra contro le forze del male. Se l'\textit{ānāpānasati} vi
suscita avversione, prendete nota anche di questo. Non vivetela come
qualcosa che dovete fare, ma come un piacere, come una esperienza che vi
piace fare. Non dovete fare nient'altro, potete rilassarvi
completamente. Avete tutto quello che vi serve, avete il respiro, dovete
solo stare seduti qui, non c'è niente di difficile da fare, non si
richiedono capacità particolari, non c'è nemmeno bisogno di essere
particolarmente intelligenti. Quando vi viene da pensare "non ci
riesco", riconoscetela come una semplice resistenza, paura o
frustrazione, e poi rilassatevi.

Se vi accorgete che la pratica dell'\textit{ānāpānasati} vi suscita particolare
tensione e preoccupazione, smettete. Non complicatela, non trasformatela
in un compito gravoso. Se non ci riuscite, limitatevi a stare seduti.
Quando mi capitava di ridurmi in questo stato, prendevo "pace" come
unico oggetto di contemplazione. Quando cominciavo a pensare: "devo...
devo... devo farcela", mi dicevo: "Stai calmo, rilassati".

Dubbi e irrequietezza, insoddisfazione, avversione - presto fui in grado
di contemplare la pace, ripetendomi la parola più e più volte, quasi
volessi ipnotizzarmi: "Rilassati, rilassati". Venivano fuori dubbi
egocentrici di ogni tipo: "così non combino niente, è inutile, voglio
dei risultati". Presto fui in grado di fare pace con queste
interferenze. Prima calmatevi, poi, quando siete rilassati, praticate
\textit{ānāpānasati}. Se cercate qualcosa da fare, fate questo.

All'inizio la pratica può essere molto noiosa; ci si sente
disperatamente impacciati, come un chitarrista alle prime armi. Quando
si comincia a suonare le dita sono goffe; sembra un'impresa impossibile,
ma poi, con l'esercizio, ci si impratichisce e diventa facile. Ora state
imparando a essere testimoni di quello che accade nella vostra mente,
così potete sapere quando state diventando irrequieti e tesi, ostili a
tutto, e ne prendete atto, non cercate di convincervi che non sia così.
Siete pienamente consapevoli di come stanno le cose: che fare quando
siete preoccupati, tesi e nervosi? Rilassarsi.

Nei miei primi anni con Ajahn Chah\footnote{Ajahn Chah è stato il maestro thailandese di Ajahn
Sumedho.} a volte prendevo la
meditazione terribilmente sul serio, e mi trattavo con una severità e
una prosopopea davvero eccessive. Perdevo del tutto il senso
dell'umorismo e diventavo mortalmente serio, rigido come un palo. Mi
sforzavo parecchio, ma era così stressante e spiacevole pensare sempre
"devo farcela... sono troppo pigro". Mi sentivo orribilmente in colpa se
non meditavo continuamente - uno stato d'animo cupo, senza gioia. Allora
cominciai a osservare questo, meditando sul mio essere rigido come un
palo di legno. Quando la situazione diventava insostenibile, mi
ricordavo degli atteggiamenti opposti: "Non sei tenuto a fare nulla. Non
devi arrivare da nessuna parte, non c'è niente da fare. Stai in pace con
le cose così come sono adesso, rilassati, lascia andare". Mi aiutavo
così.

Quando la mente cade in questo stato, applicate l'opposto, per poi
imparare a prendere le cose come vengono. Capita di leggere libri dove
si dice che non c'è bisogno di nessuno sforzo, che bisogna "lasciare che
tutto accada spontaneamente"; allora si è portati a credere che basti
starsene con le mani in mano. Il risultato di solito è scivolare in uno
stato mentale di torpore, di passività. Allora è il momento di metterci
un tantino di sforzo in più.

Nella pratica di \textit{ānāpānasati} si può sostenere lo sforzo per la durata di
un'inspirazione. E se non ci riuscite per un'intera inspirazione, fatelo
almeno per la metà. In questo modo, non cercate di diventare perfetti in
un colpo solo. Non occorre fare le cose tutte per benino in ossequio a
un'idea di come potrebbe essere; si lavora con i problemi così come si
presentano. Se però avete una mente distratta, è saggio riconoscere che
la mente se ne va per i fatti suoi: questa è “visione profonda”. Invece,
pensare che non dovreste essere distratti, odiarvi o scoraggiarvi perché
di fatto è quello che vi capita, è ignoranza.

Nell'\textit{ānāpānasati} si prende atto di come stanno le cose al momento e si
parte da lì; si sostiene l'attenzione un po' più a lungo e si comincia a
capire che cos'è la concentrazione, prendendo risoluzioni a cui si è in
grado di tener fede. Non prendete risoluzioni da superuomini quando non
siete tali. Praticate \textit{ānāpānasati} per dieci minuti o un quarto d'ora
invece di credere di poter tirare avanti tutta la notte: "Praticherò
\textit{ānāpānasati} fino all'alba". Poi non ci riuscite e vi arrabbiate. Fissate
i tempi della seduta in base alle vostre possibilità. Sperimentate,
lavorate con la vostra mente finché non capirete come sforzarvi, come
rilassarvi.

L'\textit{ānāpānasati} è una via diretta. Vi porta alla “visione profonda”, alla
\textit{vipassanā}. La natura impermanente del respiro non è vostra, no? Essendo
nato, il corpo respira spontaneamente. Inspirazione ed espirazione:
l'una condiziona l'altra. Finché il corpo vivrà sarà così. Non avete
controllo su nulla, il respiro appartiene alla natura, non appartiene a
voi, è non-sé. Osservare tutto ciò è \textit{vipassanā}, “visione profonda”. Non
è nulla di eccitante o di affascinante o di spiacevole. E' naturale.

