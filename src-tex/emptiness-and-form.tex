% Emptiness and Form

Quando la mente è tranquilla, ascoltate, sentirete una specie di ronzio;
il cosiddetto "suono del silenzio". Che cos'è? E' un suono prodotto
dall'orecchio o è un suono esterno? E' prodotto, dalla mente, dal
sistema nervoso o cosa? Sia quel che sia, è sempre lì, e si può usare in
meditazione in quanto oggetto a cui rivolgere l'attenzione.

Prendendo atto che tutto ciò che sorge passa, cominciamo ad osservare
ciò che non sorge e non passa ed è sempre presente. Se cominciate a
pensare a quel suono, a dargli un nome, a pretendere di ricavarne
qualcosa, è chiaro che lo state usando nel modo sbagliato. E' solo un
punto di riferimento a cui rivolgersi quando si giunge ai confini della
mente, a ciò che alla nostra osservazione appare come l'estremo limite
della mente. Quindi da quella posizione si può cominciare ad osservare.
Potete pensare e contemporaneamente ascoltare il suono (nel caso cioè in
cui pensiate deliberatamente); se invece vi perdete nei pensieri lo
dimenticate e non lo udite più. Quindi, se vi perdete nei pensieri, non
appena ve ne accorgete riportate l'attenzione a quel suono e ascoltatelo
a lungo. Laddove prima eravate trascinati via dalle emozioni, dalle
preoccupazioni o dagli impedimenti, ora potete praticare contemplando
con gentilezza, con molta pazienza, quella particolare condizione della
mente in quanto anicca, dukkha, anattā, e poi lasciarla andare. E' un
lasciar andare dolce, sottile, non un rifiuto violento delle condizioni.
Quindi ciò che più conta è l'atteggiamento, la retta comprensione. Non
aspettatevi nulla dal “suono del silenzio”. C'è chi si esalta, pensando
di aver raggiunto o scoperto chissà che, ma anche questa è di per sé una
condizione creata attorno al silenzio. E' una pratica molto serena, non
emozionante; usatela con abilità e dolcezza per lasciar andare,
piuttosto che per attaccarvi all'opinione di aver raggiunto qualcosa! Se
c'è un ostacolo alla meditazione, è proprio l'impressione di averne
tratto un qualche vantaggio!

Ora, riflettete sulle condizioni del corpo e della mente e concentratevi
su di esse. Potete passare in rassegna il corpo e riconoscere le
sensazioni, ad esempio le vibrazioni nelle mani o nei piedi, oppure
concentrarvi su un punto qualunque del corpo. Percepite la sensazione
della lingua in bocca che tocca il palato, o il contatto fra il labbro
superiore e quello inferiore, oppure portate alla coscienza la
sensazione di umidità della bocca, o la pressione degli indumenti sul
corpo; tutte quelle sensazioni sottili a cui in genere non facciamo
caso. Contemplando queste sottili sensazioni fisiche, concentratevi su
di esse, e il corpo si rilasserà. Al corpo umano piace ricevere
attenzione. E' contento di essere l'oggetto di una concentrazione
delicata e tranquilla, però se siete così sconsiderati da odiarlo può
darvi molto filo da torcere. Ricordate che dovete vivere in questa
struttura per il resto della vostra vita. Perciò fareste meglio a
imparare a conviverci con un atteggiamento positivo. Qualcuno dirà: "Ma
il corpo non è importante, è solo una cosa repellente, invecchia, si
ammala e muore. Il corpo non conta, l'importante è la mente". E' un
atteggiamento piuttosto diffuso fra i buddhisti! Ma in realtà ci vuole
pazienza per concentrarsi sul corpo motivati da altro che non da vanità!
La vanità è abuso del corpo umano, ma toccarlo con la consapevolezza e
sano. Non serve a rafforzare il senso dell'io, è solo un'espressione di
benevolenza nei confronti di un corpo vivente, che ad ogni modo non si
identifica con voi.

Dunque adesso la meditazione si incentra sui cinque khanda\footnote{Khanda: le cinque categorie in cui, secondo il Buddha,
andiamo ad identificare un senso del sé, ossia: rūpa, (forma, il corpo),
vedanā (sensazioni), saññā (percezione), sankhāra (formazioni mentali),
viññāna (coscienza sensoriale). In breve “il corpo e la mente”.} e
sul vuoto della mente. Investigateli entrambi fino a comprendere
pienamente che tutto ciò che sorge, passa ed è non-sé. Allora non ci si
afferra più a nulla come se fosse il sé, e ci si libera dal desiderio di
riconoscersi in una certa qualità o in una sostanza. Questa è la
liberazione dalla nascita e dalla morte.
La via della saggezza non consiste nel coltivare la concentrazione per
accedere a stati di trance, esaltarsi ed estraniarsi dalla realtà.
Bisogna essere molto onesti circa le proprie intenzioni. Meditiamo per
evadere dalla realtà? Stiamo cercando di accedere a uno stato dove tutti
i pensieri vengono annullati? Questa pratica di saggezza è un metodo non
violento in cui si lasciano emergere anche i pensieri più orribili e li
si lascia andare. Avete una via di sfogo, una specie di valvola di
sicurezza da cui far uscire il vapore quando la pressione è troppa.
Normalmente, quando si sogna molto, il vapore si scarica durante il
sonno. Ma questo non alimenta la saggezza, vi pare? E' una vita da
animali inconsapevoli: esercitare certe attività abituali, esaurire le
energie, crollare nel sonno, alzarsi, rimettersi in attività e crollare
di nuovo. Invece questo sentiero implica un'approfondita esplorazione e
una comprensione dei limiti della condizione mortale del corpo e della
mente. State sviluppando la capacità di distogliervi dalla realtà
condizionata e affrancare la vostra identità da ciò che muore.

State dissolvendo l'illusione di essere un'entità mortale, ma non vi
dico neppure che siete creature immortali, altrimenti comincereste ad
aggrapparvi a questo! "La mia vera natura è la verità ultima,
l'assoluto. Io e il Signore siamo una cosa sola. La mia vera natura è
l'Immortale, beata eternità senza tempo". Ma avrete notato che il Buddha
si guardò bene dall'usare un linguaggio poetico o ispirato. Non perché
sia sbagliato, ma perché suscita attaccamento. Perché partiremmo alla
conquista dell'identità con l'assoluto, dell'unione con Dio, della
beatitudine eterna, della dimensione immortale e via dicendo. Cose che a
dirle ci mandano in estasi. Però è molto più sano osservare la nostra
tendenza a voler definire o concepire l'inconcepibile, a volerlo
comunicare o descrivere agli altri solo per sentire di aver raggiunto
qualcosa. E' molto più importante osservarla, che seguirla. E d'altro
canto non voglio negare che una realizzazione ci sia; solo siate tanto
accorti e vigili da non attaccarvici, perché se lo fate ne ricaverete
solo nuova disperazione.

Se poi vi capitasse di farvi prendere la mano, appena ve ne accorgete,
fermatevi. Non è certo il caso di sentirsi in colpa o farne una
tragedia; basta fermarsi lì. Calmatevi, lasciate andare, lasciate
perdere la faccenda. Capita che le persone religiose abbiano delle
intuizioni che le rendono molto ispirate. I cristiani carismatici ardono
letteralmente di fervore religioso. Cosa notevole, non c'è che dire.
Devo ammettere che fa una certa impressione vedere persone così
entusiaste. Però in termini buddhisti questo stato si definisce sannā
vipallāsa, la "pazzia del meditante". Quando un insegnante esperto vi
vede in quello stato vi spedisce in una capanna nella foresta con
l'ordine di non avvicinare nessuno!

Una volta ci sono passato anch'io, quando ero a Nong Khai un anno prima
di conoscere Ajahn Chah. Me ne stavo seduto nella mia capanna convinto
di aver raggiunto l'illuminazione. Sapevo tutto, capivo tutto. Ero al
settimo cielo... peccato che non avessi nessuno con cui parlare. Non
parlavo il thailandese perciò non potevo importunare i monaci locali. Ma
un giorno passò di lì il console britannico di Vientiane e qualcuno lo
portò alla mia capanna... quella volta lo subissai davvero il poveretto!
Se ne stava lì con l'aria attonita, ed essendo inglese era anche molto,
molto, molto educato, ogni volta che faceva il gesto di alzarsi e
andarsene lo facevo rimettere seduto. Non potevo fermarmi, c'era questa
valanga di energia tipo cascate del Niagara che veniva fuori e non c'era
verso di arginarla. Alla fine il console riuscì non so come a guadagnare
l'uscita. Da allora, chissà perché, non l'ho più rivisto...

Sicché, quando passiamo per esperienze del genere, è importante
rendersene conto. Se uno sa di che si tratta, non c'è pericolo. Siate
pazienti, non date ad esse troppo credito e non vi ci adagiate. Avrete
notato che i monaci buddhisti non se ne vanno in giro a raccontare quale
"livello di illuminazione" hanno raggiunto, semplicemente, non c'è
niente da raccontare. Quando ci chiedono di insegnare, non parliamo
della “nostra” illuminazione, ma delle “Quattro Nobili Verità” in quanto
veicolo di illuminazione per tutti. Al giorno d'oggi c'è una quantità di
personaggi che dicono di essere illuminati, Buddha Maitreya o un qualche
avatar, e tutti hanno schiere di seguaci. La gente è disposta a crederci
senza troppa difficoltà! Ma il Buddha sottolineava l'importanza di
riconoscere le cose così come sono, invece di credere a quello che ci
raccontano gli altri. La nostra è una via di saggezza, nella quale
esploriamo e indaghiamo, i limiti della mente. Constatatelo di persona:
sabbe sankhārā aniccā, tutti i fenomeni condizionati sono impermanenti;
sabbe dhammā anattā, tutte le cose sono non-sé.
