
Praticando l'ascolto interiore, cominciamo a riconoscere il sussurro del
senso di colpa, del rimorso e del desiderio, dell'invidia e della paura,
della brama e dell'avidità. A volte ascolterete cosa dice la brama:
"Vorrei, devo avere, devo avere, voglio, voglio!". A volte non ha
neppure un oggetto preciso. Può esserci una brama senza oggetto, per cui
le troviamo un oggetto. Il desiderio di ottenere: "Voglio qualcosa,
voglio qualcosa! Devo averlo, voglio...". Ascoltate la mente, forse lo
sentirete. Di solito riusciamo a trovare un oggetto per la nostra brama,
per esempio il sesso, oppure passiamo il tempo a fantasticare.

La brama può prendere la forma del cercare qualcosa da mangiare, o in
cui immergersi completamente, del diventare qualcosa, unirsi a qualcosa.
La brama sta sempre di vedetta, sempre alla ricerca di un oggetto. Può
essere un oggetto gradevole permesso ai monaci, come una bella veste o
una ciotola o un cibo saporito. Notate la tendenza a volerlo, a
toccarlo, cercare di procurarselo, averlo, possederlo, appropriarsene,
consumarlo. E' la brama, una forza della natura che va riconosciuta; non
condannata pensando "sono un individuo spregevole perché desidero",
perché anche questo fa da rinforzo all'io, vi pare? Come se dovessimo
essere completamente esenti da brama, come se esistessero esseri umani
che non fanno esperienza del desiderio!

Sono condizioni naturali che dobbiamo riconoscere e notare; non per
condannarle, ma per comprenderle. Così cominciamo veramente a conoscere
il movimento mentale della brama, dell'avidità, del ricercare qualcosa;
come pure del desiderio di farla finita con qualcosa. Anche di questo si
può essere testimoni, del desiderio di sbarazzarci di quello che
abbiamo, di una situazione o anche del dolore. "Voglio farla finita con
questo dolore, con le mie debolezze, con il torpore, con la mia
irrequietezza, con la mia brama. Voglio liberarmi di tutto ciò che mi dà
fastidio. Perché Dio ha creato le zanzare? Voglio liberarmi dalle
seccature".

Il desiderio sensoriale è il primo degli impedimenti (nīvarana). Il
secondo è l'avversione; la mente è ossessionata dal non volere, da
irritazioni e risentimenti meschini, nonché dal desiderio di eliminarli.
Quindi questo è un ostacolo alla visione interiore, è un impedimento.
Non sto dicendo che bisogna eliminare l'impedimento - sarebbe avversione
- ma che bisogna conoscerlo, conoscerne la forza, comprenderlo per
esperienza diretta. Allora si prende coscienza del desiderio di
sbarazzarsi di cose che sono dentro di sé, o attorno a sé, del desiderio
di non esserci, di non essere vivi, di non esistere più. E' per questo
che ci piace dormire, no? Perché per un po' ci consente di non esistere.
Nella coscienza caratteristica del sonno non esistiamo, perché la
sensazione stessa di essere vivi viene a mancare. C'è un annullamento.
E' per questo che alcune persone dormono molto, perché vivere è troppo
doloroso per loro, troppo noioso, troppo sgradevole. Quando siamo
depressi, dubbiosi, disperati, cerchiamo scampo nel sonno, cerchiamo di
annullare i nostri problemi, estromettendoli dalla coscienza.

Il terzo impedimento è rappresentato da stati come sonnolenza, apatia,
ottundimento, indolenza, torpore fisico e mentale, ai quali tendiamo a
reagire con avversione. Ma è sempre qualcosa che può essere compreso.
L'opacità può essere conosciuta, la pesantezza fisica e mentale, il
movimento lento, opaco. Osservate l'avversione per questi stati, il
desiderio di sbarazzarvene. Osservate la sensazione di opacità nel corpo
e nella mente. Anche la conoscenza del torpore è mutevole, è
insoddisfacente, impersonale.

L'irrequietezza è l'opposto del torpore; è il quarto impedimento. Non si
è affatto opachi, né sonnolenti, ma viceversa agitati, nervosi, ansiosi,
tesi. Anche qui può non esserci un oggetto specifico. Diversamente dalla
sonnolenza, l'irrequietezza è uno stato più ossessivo. Si vorrebbe
essere attivi, correre, fare questo, fare quello, parlare, andare in
giro, agitarsi. E se dovete stare seduti immobili per un po' quando vi
sentite irrequieti vi sentite in trappola, chiusi in una gabbia; non
pensate ad altro che a saltare, correre in giro, darvi da fare. Anche di
questo si può essere consapevoli, specialmente quando si è contenuti da
una forma che non ci consente di seguire l'irrequietezza. L'abito che
indossano i bhikkhu non è particolarmente adatto per arrampicarsi sugli
alberi e penzolare dai rami. Non potendo agire questa tendenza salterina
della mente siamo costretti a osservarla.

Il quinto impedimento è il dubbio. A volte i nostri dubbi ci sembrano
importantissimi, e ci piace dar loro parecchia attenzione. Siamo
facilmente ingannati dalla natura del dubbio, perché sembra molto reale:
"Certi dubbi sono futili, è vero, ma questo è un Dubbio Importante. Devo
sapere la risposta. Devo essere sicuro. Devo saperlo assolutamente:
meglio fare questo oppure quest'altro? Sto facendo bene? Dovrei
andarmene o restare un altro po'? Sto sprecando il mio tempo? Ho
sprecato la mia vita? Il Buddhismo è la via giusta oppure no? Forse non
è la religione giusta!". Questo è il dubbio. Si può passare tutta la
vita a chiedersi se sia meglio fare questo o quello, ma una cosa sola si
può sapere: che il dubbio è una condizione della mente. A volte prende
forme sottili e ingannatrici. Assumendo la posizione del "conoscitore",
conosciamo il dubbio in quanto tale. Importante o futile che sia, è
semplicemente dubbio, tutto qui. "Devo restare qui o andarmene
altrove?": è un dubbio. "Devo fare il bucato oggi o domani?": è un
dubbio. Non importantissimo, ma poi ci sono quelli importanti: "Sono già
un sotapanna?\footnote{Sotapanna: un praticante buddhista che ha raggiunto il
primo livello di illuminazione.} Ma cos'è in definitiva un sotapanna? Ajahn
Sumedho è un arahat, un illuminato? Esistono ancora arahant al giorno
d'oggi?". Poi i seguaci di altre religioni vengono a dirci che la nostra
è sbagliata e la loro è quella giusta. "Forse hanno ragione! Forse siamo
in errore". Ciò che possiamo sapere è che c'è il dubbio. In questo modo
siamo il conoscere, conoscere ciò che si può conoscere, sapere che non
sappiamo. Anche quando ignorate qualcosa, se siete consapevoli di non
sapere, quella consapevolezza è conoscenza.
Sicché, “essere il conoscere” significa questo, conoscere ciò che si può
conoscere. I cinque impedimenti sono i vostri maestri, perché non sono i
guru esaltanti e radiosi che si vedono sui libri. Possono essere
piuttosto volgari, meschini, sciocchi, irritanti e ossessionanti. E ci
incalzano, ci punzecchiano, ci riducono a mal partito, finché non gli
diamo l'attenzione e la comprensione dovuta, finché non smettono di
essere problemi. Ecco perché bisogna essere molto pazienti; dobbiamo
avere tutta la pazienza del mondo, e l'umiltà di imparare dai nostri
cinque maestri.

E che cosa impariamo? Che non sono altro che condizioni della mente, che
sorgono e passano, che sono insoddisfacenti, impersonali. A volte si
ricevono messaggi importanti nella vita. Tendiamo a dare credito a
questi messaggi, ma ciò che possiamo sapere è che sono condizioni
mutevoli: e se abbiamo la pazienza di tollerarle fino in fondo le cose
cambiano automaticamente, per conto loro, e noi abbiamo l'apertura e la
chiarezza mentale per agire spontaneamente, invece di reagire alle
condizioni.

Grazie alla nuda attenzione, alla consapevolezza, le cose fanno il loro
corso, non c'è bisogno di sbarazzarsene, perché tutto ciò che ha
principio, ha fine. Non c'è nulla da eliminare, bisogna solo essere
pazienti e lasciare che tutto faccia il suo corso naturale verso la
cessazione.

Quando siete pazienti, lasciando che le cose cessino, cominciate a
conoscere la cessazione - il silenzio, il vuoto, la chiarezza: la mente
è limpida, quieta, ed è ancora vibrante, non cade nell'oblio, non è
repressa o addormentata, e si può udire il silenzio della mente.

Consentire la cessazione significa essere molto gentili, molto delicati
e pazienti, umili, senza schierarsi dalla parte di nulla: del bene, del
male, del piacere o del dolore. L'accettazione gentile permette alle
cose di cambiare secondo la propria natura, senza interferenze. Allora
impariamo a smettere di cercare di assorbirci negli oggetti dei sensi.
Troviamo la pace nel vuoto della mente, nella sua chiarezza, nel suo
silenzio.

