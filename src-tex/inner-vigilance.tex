
Passiamo ora alla pratica della consapevolezza. La concentrazione
consiste come abbiamo visto nel rivolgere l'attenzione a un determinato
oggetto e nel mantenerla focalizzata su di esso (ad esempio il ritmo
tranquillizzante della normale respirazione) finché ci si immedesima con
quel segnale e la percezione di un soggetto e un oggetto separati sfuma.
Nella meditazione \textit{vipassanā}, la consapevolezza ha invece a che vedere
con una mente aperta. Non ci si concentra più su un solo oggetto ma si
osserva in profondità, contemplando i fenomeni condizionati che vanno e
vengono e il silenzio della mente vuota. Per far questo bisogna lasciare
andare l'oggetto; non ci si aggrappa a un oggetto particolare, ma si
osserva che tutto ciò che sorge svanisce. Questa è la meditazione di
consapevolezza, o \textit{vipassanā}.

Nella pratica che definisco "ascolto interiore", si ascoltano i rumori
che emergono nella mente: il desiderio, le paure, i contenuti repressi a
cui non è stato mai permesso di emergere pienamente alla coscienza.
Adesso però, anche se affiorano pensieri ossessivi, timori o altre
emozioni, siate disposti a farli emergere alla coscienza e a lasciarli
cessare spontaneamente. Se non c'è nulla che viene e va, dimorate
semplicemente nel vuoto, nel silenzio della mente. Potrete udire una
vibrazione ad alta frequenza nella mente, che è sempre presente e non è
un suono prodotto dall'orecchio. Quando lasciate andare le condizioni
della mente potete rivolgere l'attenzione a quel suono. Ma riconoscete
onestamente le vostre intenzioni. Tanto che, se vi rivolgete al
silenzio, al suono silenzioso della mente, spinti dall'avversione per le
condizioni, state ancora una volta reprimendo, non è un processo di
purificazione.

Se l'intenzione è scorretta, anche se vi concentrate sul vuoto non
avrete buoni risultati perché siete andati fuori strada. Non avete
riflettuto saggiamente, non avete lasciato andare nulla, vi state solo
ritraendo per avversione. E' come se uno dicesse: "Non voglio vedere" e
si voltasse dall'altra parte.

Si tratta quindi di una pratica paziente in cui siamo disposti a
tollerare ciò che sembra intollerabile. E' uno stato di vigilanza
interiore, è osservare, ascoltare, sperimentare. In questa pratica
quello che conta è la “retta comprensione”, più che il vuoto, la forma o
altre cose del genere. La “retta comprensione” nasce dall'aver visto che
tutto ciò che sorge passa; dall'aver visto che anche il vuoto non è il
sé. Affermare di aver realizzato il vuoto come una specie di conquista,
già di per sé è un'intenzione scorretta, vi pare? Credere di essere
qualcuno che ha ottenuto una certa realizzazione personale deriva dal
senso dell'io. Perciò non affermiamo nulla. Se c'è qualcosa in voi che
desidera farlo, allora osservate questo come una condizione della mente.

Il “suono del silenzio” è sempre presente, quindi potete usarlo come
punto di riferimento, piuttosto che come fine a se stesso. Dunque è una
pratica molto sottile di osservazione e di ascolto, non un modo per
reprimere le condizioni sull'onda dell'avversione. Poi in definitiva il
vuoto è piuttosto noioso. Siamo abituati a cose ben più stimolanti. E
comunque, per quanto tempo potete starvene seduti a osservare una mente
vuota? Quindi rendetevi conto che la nostra pratica non consiste
nell'attaccarsi alla quiete o al silenzio o al vuoto in quanto fine a se
stessi, ma nell'usarli come un abile mezzo per poter essere il
“conoscere", essere svegli. Quando la mente è vuota, osservate: la
coscienza c'è ancora, ma non tendete più a "rinascere" in questa o
quella condizione perché il senso dell'io è assente. L'io si associa
sempre all'attività di cercare qualcosa o volersi liberare di qualcosa.
Ascoltate l'io che dice: "Voglio raggiungere il \textit{samādhi}, devo
raggiungere i \textit{jhāna}". E' la voce dell'io: "Innanzitutto devo raggiungere
il primo \textit{jhāna}, o il secondo \textit{jhāna}". La solita idea che prima c'è
qualcosa da raggiungere. Quando leggete gli insegnamenti dei vari
maestri, cosa potete conoscere? Potete conoscere quando siete confusi,
quando dubitate, quando provate avversione e sospetto. Potete conoscere
che in quel momento siete il conoscere, invece di decidere quale maestro
fa al caso vostro.

Praticare \textit{mettā} significa esprimere gentilezza attraverso la capacità di
tollerare ciò che potrebbe sembrarci intollerabile. Se la mente è
ossessionata e non la smette più di ciarlare e lamentarsi, se il vostro
unico desiderio è sbarazzarvi di quella ossessione, più vi sforzate di
reprimerla e sbarazzarvene peggio è. A volte la smette, e allora: "E'
finita, me ne sono liberato". Ma poi ricomincia: "Oh no! Credevo che
fosse finita!". Perciò, dovesse pure smettere e ricominciare mille
volte, prendetela come viene. Con l'atteggiamento di chi fa un passo
alla volta. Quando vi mettete nell'ordine di idee di avere tutta la
pazienza del mondo per stare con le condizioni del momento, potete
lasciarle cessare. Il risultato del permettere alle cose di cessare è
che si comincia a sperimentare un senso di liberazione, perché ci si
rende conto di non portarsi più appresso le solite cose. Quello che un
tempo vi faceva arrabbiare, ora, con vostra sorpresa, non vi
infastidisce più di tanto. Cominciate a sentirvi a vostro agio in
situazioni che prima vi mettevano regolarmente a disagio, perché adesso
permettete alle cose di cessare, invece di tenervele strette ricreando
così paure e ansie. Anche il disagio degli altri non vi influenza. Non
reagite più al disagio degli altri irrigidendovi a vostra volta. E'
l'effetto del lasciar andare e consentire alle cose di cessare.

Dunque l'idea generale è di conservare questo stato di vigilanza
interiore, notando i contenuti che tendono a emergere ossessivamente. Se
tendono a ripresentarsi puntualmente, è segno evidente di attaccamento,
nella forma o di avversione o di infatuazione. Quindi potete cominciare
a prendere coscienza dell'attaccamento, invece di tentare di eliminarlo.
Una volta che l'avete capito e siete in grado lasciar andare potete
rivolgervi al silenzio della mente, perché non ha senso fare altro. Non
ha senso afferrarsi o aggrapparsi alle condizioni più del necessario.
Lasciatele cessare. Quando reagiamo ai contenuti che emergono,
inneschiamo un circolo vizioso abituale. Un'abitudine è qualcosa di
circolare che tende a perpetuarsi, che non ha modo di cessare. Ma se
mollate la presa e lasciate le cose a se stesse, quello che sorge cessa.
Non diventa un circolo vizioso.

Dunque il vuoto non è sbarazzarsi di tutto; non è un annullamento totale
ma infinita potenzialità creativa che appare e scompare senza che voi ne
restiate ingannati. L'idea di me stesso in quanto creatore, dei miei
talenti artistici, della mia espressione personale, è una fissazione
incredibilmente egocentrica, vi pare? "Ecco cosa ho fatto, è opera mia".
E gli altri: "Ma che talento straordinario, un vero genio!". Eppure
tanta della cosiddetta arte creativa non è che il rigurgito delle paure
e dei desideri dell'autore. Non è veramente creativa; è riproduttiva.
Non scaturisce da una mente vuota, ma da un io che non ha un reale
messaggio da dare se non il suo contenuto di morte e di egoismo. In una
prospettiva universale il suo unico messaggio è: "Guardatemi!", in
quanto persona, in quanto “io". Eppure la mente vuota ha un infinito
potenziale creativo. Non si pensa di creare nulla; ma la creazione può
avvenire senza "io" e senza nessuno che la faccia: accade.

Dunque lasciamo la creazione al \textit{Dhamma}, invece di assumercene la
paternità. L'unico compito che ci spetta, la sola necessità per noi -
sul piano convenzionale, come esseri umani, come persone - è lasciar
andare, non attaccarci. Mollare la presa. Fare il bene, non fare il
male, essere consapevoli. Un messaggio molto semplice.

