
Quando parliamo di "amore" spesso ci riferiamo a una cosa che ci piace.
Per esempio, "amo il riso al vapore", "amo il mango sciroppato". In
realtà intendiamo dire che ci piacciono, ossia che siamo attaccati a
qualcosa - come ad esempio un cibo - che ci piace molto o che gradiamo
mangiare. Ma non è che lo amiamo. \textit{Mettā} significa amare il proprio
nemico; non significa che il nostro nemico ci è simpatico. Se uno
volesse ammazzarmi, sarebbe assurdo dire che mi è simpatico! Però posso
amarlo, nel senso che posso astenermi da pensieri malevoli e
vendicativi, dal desiderio di danneggiarlo o annientarlo. Anche se una
persona non vi piace - mettiamo che sia un disgraziato, un miserabile -
potete sempre trattarla con gentilezza, generosità e carità. Se entrasse
in questa stanza un ubriaco, sporco e repellente, brutto e malandato,
privo della benché minima attrattiva, sarebbe ridicolo dire "mi piace".
Però sarebbe possibile amarlo, non indugiare nell'avversione, non farsi
coinvolgere dalle reazioni alla sua sgradevolezza. Ecco cosa intendiamo
con \textit{mettā}.

A volte ci sono aspetti di noi stessi che non ci piacciono, però \textit{mettā}
significa non farsi coinvolgere dai nostri pensieri, dagli
atteggiamenti, dai problemi e sentimenti della mente. Quindi è una
pratica di consapevolezza molto diretta. Essere consapevoli significa
avere \textit{mettā} per la paura che abita la mente, o per la rabbia, o per
l'invidia. \textit{Mettā} significa non creare problemi attorno alle condizioni
esistenti, lasciare che si dissolvano, che cessino. Per esempio, quando
emerge la paura si può avere \textit{mettā} per la paura, nel senso che non si
alimenta l'avversione nei suoi confronti, si può accettarne la presenza
e consentirle di cessare. Si può anche ridimensionarla riconoscendo che
è la medesima paura che hanno tutti, compresi gli animali. Non è la mia
paura, non è di nessuno, è una paura impersonale. Cominciamo ad avere
compassione degli altri esseri quando comprendiamo la sofferenza
implicita nel reagire alla paura nella nostra stessa vita; il dolore, il
dolore fisico dell'essere presi a calci, quando qualcuno ci prende a
calci. Quel dolore è esattamente identico a quello che prova un cane
quando viene preso a calci, perciò si può avere \textit{mettā} per il dolore,
ossia la gentilezza e la pazienza di non indulgere nell'avversione.
Possiamo rivolgere \textit{mettā} a noi stessi, ai nostri problemi emotivi; a
volte si pensa: "devo sbarazzarmene, è tremendo". Questa è mancanza di
\textit{mettā} per se stessi, vi pare? Riconoscete questo desiderio di
"sbarazzarvi di". Non nutrite avversione per le condizioni emotive
esistenti. Non si tratta di simulare approvazione per i propri difetti.
O di pensare "i miei difetti mi piacciono". Certe persone sono così
sciocche da dire: "I miei difetti mi rendono interessante. Ho una
personalità affascinante grazie alle mie debolezze". \textit{Mettā} non è
condizionarsi a credere di apprezzare qualcosa che non si apprezza
affatto, è semplicemente non nutrire avversione. E' facile provare \textit{mettā}
verso ciò che ci piace - bambini carini, belle persone, gente educata,
cuccioli e fiorellini - e possiamo provare \textit{mettā} per noi stessi quando
siamo di buon umore. "Sono proprio contento di me". Quando tutto va bene
e facile essere gentili verso ciò che è buono e grazioso e bello. Ecco
dove ci perdiamo. \textit{Mettā} non è solo questione di buone intenzioni, buoni
sentimenti, pensieri elevati, è sempre estremamente pratica.

Se siete molto idealistici e odiate qualcuno, è facile che
l'atteggiamento sia: "Non devo odiare nessuno. I buddhisti devono
nutrire \textit{mettā} per tutti gli esseri viventi. Dovrei amare tutti". Che è
solo frutto di astratto idealismo. Nutrire \textit{mettā} per l'avversione che si
prova, per le meschinità della mente, la gelosia, l'invidia; nel senso
di convivere pacificamente, non creare problemi, non drammatizzare e
complicare le difficoltà che la vita ci pone, a livello fisico e
mentale.

Quando vivevo a Londra, viaggiare in metropolitana mi infastidiva
moltissimo. Le odiavo, quelle orribili stazioni della metro tappezzate
di squallidi cartelloni pubblicitari e quella massa di gente stipata su
quei brutti treni anneriti che urlano nelle gallerie. Provavo una totale
mancanza di \textit{mettā}, di pazienza amorevole. Prima provavo solo avversione,
ma poi decisi di praticare la gentilezza paziente ogni volta che
prendevo la metropolitana di Londra. Alla fine cominciai a prenderci
gusto, invece di indulgere al risentimento. Cominciai a sentirmi
disponibile verso gli altri passeggeri. L'avversione e le lamentele
svanirono, completamente.

Quando provate avversione per qualcuno, potete notare la tendenza a
ricamarci sopra: "Ha fatto questo e quest'altro, è fatto così e non
dovrebbe essere in tal modo". Invece se una persona vi e simpatica: "Sa
fare questo e quest'altro. E' buono e gentile". Poi se qualcuno insinua
che è un poco di buono, vi arrabbiate. Se odiate qualcuno e un altro lo
loda, vi arrabbiate anche in quel caso. Non volete sentire quanto è
buono il vostro nemico. Quando siete in preda alla rabbia non riuscite a
immaginare che la persona odiata possa avere delle buone qualità, e se
anche le ha non ve ne viene in mente nessuna. Ricordate solo i lati
cattivi. Ma se qualcuno vi piace anche i suoi difetti - quei "difettucci
innocenti" - vi fanno simpatia.

Sicché, prendetene atto nella vostra esperienza personale; osservate la
forza delle vostra simpatie e antipatie. La gentilezza paziente, \textit{mettā},
è uno strumento molto utile ed efficace per lavorare con il cumulo di
meschine banalità che la mente imbastisce attorno a un'esperienza
spiacevole. \textit{Mettā} è inoltre un metodo assai adatto a chi possiede una
mente discriminante, ipercritica. A chi sa vedere solo il lato negativo
delle cose ma non guarda mai se stesso, vede soltanto quello che è fuori
di lui.

E' un'abitudine ormai diffusa quella di lamentarsi sempre del governo o
del tempo. L'arroganza genera la ben nota tendenza alla battuta pesante
su questo e quello; oppure il fatto di sparlare degli assenti, facendoli
a pezzi con straordinaria intelligenza e obiettività. Con quanto acume,
con quanta precisione sappiamo dire cosa servirebbe agli altri, cosa
dovrebbero o non dovrebbero fare, perché sono fatti così e colà. Che
cosa straordinaria, possedere uno spirito critico così raffinato e saper
dire agli altri quello che dovrebbero fare. In realtà stiamo dicendo:
"Chiaramente sono molto meglio io di loro".

Il punto non è rendersi ciechi alle magagne e ai difetti delle
situazioni. Si tratta di saperci convivere pacificamente. Senza
pretendere che siano altrimenti. Sicché \textit{mettā} a volte richiede di
passare sopra ai lati negativi di noi stessi e degli altri; il che non
significa non accorgersi che ci sono, significa non farsene un problema.
Si mette un freno a questa sorta di vizio essendo gentili e pazienti,
convivendo pacificamente.

