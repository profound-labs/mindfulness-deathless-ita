
Siamo qui accomunati da un unico interesse. Invece di essere un gruppo
di individui in cui ciascuno segue le proprie opinioni e i propri punti
di vista, stasera ci ritroviamo qui motivati dal comune interesse per la
pratica del \textit{Dhamma}. Quando un così gran numero di persone si trova
riunito una domenica sera, si comincia a intravedere il potenziale
insito nell'esistenza umana, una società basata sul comune interesse per
la verità. Nel \textit{Dhamma} ci uniamo. Ciò che sorge passa, e ciò che resta è
pace. Sicché, quando cominciamo ad abbandonare le abitudini e
l'attaccamento ai fenomeni condizionati, cominciamo a riconoscere
l'integrità e l'unità della mente.

E' una riflessione molto importante per l'epoca in cui viviamo, così
segnata dai conflitti e dalle guerre che nascono quando non si riesce ad
andare d'accordo su nulla. I Cinesi contro i Russi, gli Americani contro
i Sovietici, e via di questo passo. perché? Qual è il motivo del
contrasto? Sono le rispettive percezioni del mondo. ``Questa è la mia
terra e la voglio così. Voglio questa forma di governo e questo sistema
politico ed economico'', e via di tale passo. Finché si arriva
all'assassinio e alla tortura e perfino ad annientare il paese che
vorremmo liberare, a schiavizzare o manipolare il popolo che vorremmo
libero. Perché? perché non comprendiamo la vera natura delle cose.

La via del \textit{Dhamma} consiste nell'osservare la natura ed armonizzare la
nostra vita con le forze naturali. Nella civiltà europea siamo ben
lontani dal guardare il mondo in questi termini. Lo abbiamo idealizzato.
Se tutto andasse secondo i nostri ideali, dovrebbe essere in un certo
modo. E quando ci attacchiamo agli ideali finiamo col fare quello che
abbiamo fatto al nostro pianeta, contaminato e trascinato sull'orlo
della distruzione totale perché non comprendiamo i limiti che ci impone
la vita sulla terra. Sicché, da questo punto di vista, a volte ci tocca
imparare la dura lezione facendo errori e combinando un sacco di guai.
Auspicabilmente, non è una situazione irreparabile.

Ora, in questo monastero i monaci e le monache praticano il \textit{Dhamma} con
diligenza. Per tutto il mese di gennaio non parliamo neppure, ma
dedichiamo le nostre vite e offriamo i meriti della nostra pratica per
il bene di tutti gli esseri senzienti. L'intero mese è una preghiera
incessante, un'offerta di questa comunità per il bene di tutti gli
esseri senzienti. E' un periodo interamente dedicato alla ricerca della
verità, a osservare e ascoltare e guardare le cose così come sono; un
periodo in cui ci si astiene dall'indulgere alle abitudini e agli stati
d'animo egocentrici, rinunciando a tutto questo per il bene degli esseri
viventi. E' una testimonianza offerta a tutti perché riflettano sulla
dedizione e il sacrificio che il cammino verso la verità comportano. E'
un invito a realizzare la verità nella propria vita, invece di vivere in
modo meccanico e abitudinario, assecondando le condizioni del momento.
E' una riflessione per gli altri. Abbandonare i comportamenti immorali,
egoistici o violenti per essere persone che aspirano alla virtù, alla
generosità, alla moralità e all'azione compassionevole nel mondo. Se non
facciamo questo, allora la nostra situazione è assolutamente senza
speranza. Tanto varrebbe che facessimo saltare per aria tutto, perché se
nessuno è disposto a usare la propria vita per qualcosa di più che il
proprio egoismo, è tutto inutile.

Questo paese è un paese generoso e benevolo, ma noi lo diamo per
scontato e lo sfruttiamo per quel che ne possiamo ricavare. Non pensiamo
granché a quel che potremmo offrirgli. Esigiamo molto, vogliamo che il
governo ci risolva tutti i problemi salvo poi criticarlo se non ci
riesce. Ai nostri giorni vediamo individui egoisti che vivono a modo
loro, senza riflettere saggiamente e adottare uno stile di vita
vantaggioso per la collettività nel suo insieme. In quanto esseri umani
possiamo fare della nostra vita una grande benedizione, o diventare un
cancro del pianeta, sfruttando le risorse della terra per il nostro
personale guadagno e accaparrando il più possibile per ``me'' e per il
``mio''.

Nella pratica del \textit{Dhamma} il senso del me e del mio comincia a sbiadire,
quel senso dell'io-mio in quanto piccola creatura seduta qui che ha una
bocca e deve mangiare. Se non faccio altro che seguire i desideri del
mio corpo e le mie emozioni, non sarò che una piccola creatura avida ed
egoista. Ma quando rifletto sulla natura della mia condizione fisica e
su come può essere usata abilmente in questo spazio di vita per il bene
di tutti, allora questo stesso essere diventa una benedizione (tuttavia
non è che si pensa ``sono una benedizione''; attaccarsi all'idea di essere
una benedizione è un'altra forma di orgoglio!). Sicché si tratta di
vivere giorno per giorno in modo da esprimere gioia, compassione,
gentilezza attraverso la propria vita, o quanto meno in modo tale da non
causare inutile confusione e dolore. Il minimo che possiamo fare è
osservare i ``cinque precetti'' affinché il nostro corpo e le
nostre parole non divengano strumento di violenza, crudeltà e
sfruttamento nei confronti del pianeta. Vi sto forse chiedendo troppo?
E' irrealistico che io rinunci a fare semplicemente quel che mi pare al
momento per essere almeno un pochino più attento e responsabile in ciò
che faccio e che dico? Tutti possiamo cercare di essere d'aiuto, essere
generosi e gentili e rispettosi nei confronti degli altri esseri con cui
ci troviamo a condividere il pianeta. Tutti possiamo interrogarci con
saggezza per arrivare a comprendere le limitazioni a cui siamo
sottoposti, in modo da non farci più ingannare dal mondo sensoriale.
Ecco perché meditiamo. Per un monaco o una monaca è uno stile di vita,
un sacrificio dei nostri desideri e capricci particolari per il bene
della comunità, del \textit{Sangha}.

Se mi metto a pensare a me stesso e a cosa voglio io, è facile che mi
dimentichi di voi, perché ciò che voglio io in un dato momento può non
andare bene per tutti quanti gli altri. Ma quando prendo come guida il
mio rifugio nel \textit{Sangha}, allora il benessere del \textit{Sangha} è la mia gioia e
posso rinunciare ai miei capricci per il bene del \textit{Sangha}. Ecco perché i
monaci e le monache si rasano la testa e vivono sotto la disciplina
stabilita dal \textit{Buddha}. E' un modo per educarsi al lasciare andare l'io
come modo di vivere: un modo di vivere in cui vergogna, senso di colpa e
paura non hanno più ragione di essere. Si perde la sensazione di
un'individualità aggressiva, perché non si è più tesi a considerarsi
indipendenti dal resto o a dominare, ma a vivere in armonia per il bene
di tutti gli esseri, piuttosto che per il proprio bene.

La comunità dei laici ha l'opportunità di partecipare a questo. I monaci
e le monache dipendono dai laici per il loro sostentamento, quindi è
importante per la comunità dei laici assumersene la responsabilità. E'
un modo di uscire dai vostri problemi e dalle vostre preoccupazioni
particolari, perché quando vi date il tempo di venire qui per donare,
per aiutare, per praticare la meditazione e ascoltare il \textit{Dhamma}, ci
ritroviamo insieme nell'unità della verità. Possiamo essere qui insieme
senza invidia, gelosia, paura, dubbio, avidità o desiderio grazie alla
nostra inclinazione verso la ricerca della verità. Fate che sia questa
l'intenzione portante della vostra vita, non sprecatela inseguendo mete
senza valore!

Questa verità si può chiamare in molti modi. Le religioni si sforzano di
comunicarla con certi mezzi - attraverso concetti e dottrine - ma noi
abbiamo dimenticato che cos'è la religione. In questi ultimi secoli la
nostra società ha visto il predominio della scienza materialistica, del
pensiero razionale e dell'idealismo basato sulla nostra capacità di
concepire sistemi economici e politici; eppure non riusciamo a farli
funzionare, è vero? Non riusciamo a creare una vera democrazia, o un
vero comunismo o un vero socialismo; non riusciamo a crearli perché
siamo ancora illusi dal senso dell'io. Perciò tutto naufraga nelle
tirannide e nell'egoismo, nella paura e nel sospetto. Sicché l'attuale
situazione mondiale è il risultato del non aver compreso la realtà delle
cose, e d'altro canto è un'occasione per ciascuno di noi, se veramente
siamo interessati a capire cosa si può fare, per fare della propria vita
qualcosa che ha valore. Ora, in che modo possiamo farlo?

Per prima cosa bisogna prendere atto delle motivazioni ed inclinazioni
egocentriche dovute all'immaturità emotiva, per poterle conoscere ed
essere in grado di lasciarle andare; aprire la mente alle cose così come
sono, essere vigili. La nostra pratica di \textit{ānāpānasati} è un inizio, no?
Non è un'ennesima abitudine o un passatempo che coltiviamo per tenerci
occupati, ma un mezzo per sforzarci di osservare, di concentrarci ed
essere con la realtà del respiro. In alternativa si potrebbe passare un
sacco di tempo davanti alla televisione, al bar o impegnati in attività
non molto salutari; in un certo senso sembra più importante che passare
del tempo stando seduti a osservare il respiro, vero? Guardate la TV e
vedete delle persone assassinate in Libano; sembra più importante che
stare semplicemente seduti a guardare un'inspirazione e un'espirazione.
Ma questa è la mente che non comprende la realtà del cose; per cui siamo
disposti a guardare delle ombre sullo schermo e la miseria che passa
attraverso uno schermo televisivo, il dramma dell'avidità, dell'odio e
della stupidità che si perpetua nei modi più inaccettabili. Non sarebbe
molto più sano dedicare quel tempo a essere in contatto con il corpo
così com'è adesso? Sarebbe meglio nutrire rispetto per questo essere
fisico qui, e imparare a non sfruttarlo, a non abusarne, per poi
prendercela con lui quando non ci dà la felicità desiderata.

In monastero non abbiamo il televisore perché dedichiamo la nostra vita
a fare cose più utili, come osservare il nostro respiro e camminare
avanti e indietro su un sentiero nella foresta. I vicini pensano che
siamo matti. Tutti i giorni vedono uscire gente avvolta in coperte che
cammina su e giù. ``Ma che fanno? Saranno mica matti!''. Un paio di
settimane fa c'è stata una battuta di caccia alla volpe qui da noi. Mute
di cani sguinzagliati dietro le volpi della nostra foresta (cosa
veramente utile e benefica per tutti gli esseri senzienti!). Sessanta
cani e tutti quegli adulti alle calcagna di una povera volpicella! Ma
non sarebbe meglio passare il tempo a camminare avanti e indietro per la
foresta? Meglio per la volpe, per i cani, per Hammer Wood e per i
cacciatori. Ma nel West Sussex si pensa che sia normale. Loro sono i
normali e noi i pazzi. Quando osserviamo il nostro respiro e camminiamo
avanti e indietro per la foresta, se non altro non terrorizziamo le
volpi! Come vi sentireste voi ad essere inseguiti da sessanta cani?
Immaginate in che stato sarebbe il vostro cuore se aveste sessanta cani
alla calcagna e della gente a cavallo che ve li aizza contro! E' una
cosa molto brutta, se appena ci si pensa. Eppure da queste parti lo si
considera normale, o perfino desiderabile. Dato che non ci si dà la pena
di riflettere, si può diventare vittime dell'abitudine, schiavi del
desiderio e delle abitudini. Se analizzassimo in cosa consiste
effettivamente la caccia alla volpe, ce ne asterremmo. Chiunque abbia un
minimo di intelligenza e consideri seriamente la questione non sente il
desiderio di farlo. Invece, attività semplici come camminare su e giù
per un sentiero e osservare il respiro ci permettono di cominciare a
essere più consapevoli e molto più sensibili. La verità comincia a
rivelarcisi attraverso queste nostre pratiche semplici e apparentemente
insignificanti. Come del resto avviene quando osserviamo i ``cinque
precetti'', che sono una fonte di felicità per il mondo.

Quando cominciate a riflettere sulla realtà delle cose e ripensate a una
circostanza in cui la vostra vita è stata seriamente in pericolo, sapete
quanto sia orribile. E' un'esperienza assolutamente spaventosa. Chi ha
avuto modo di rifletterci, non ha la minima voglia di sottoporre
intenzionalmente un'altra creatura alla stessa esperienza. In nessuna
circostanza potrebbe far patire a un altro un simile terrore.
Diversamente, penserete che le volpi non hanno importanza, o che i pesci
non hanno importanza. Esistono solo per il mio piacere, un passatempo
per la domenica pomeriggio. Ricordo una donna che venne a trovarmi ed
era molto seccata che fossimo noi i nuovi proprietari del laghetto di
Hammer Wood.\footnote{Essendo parte di un monastero buddhista, sia la foresta
che il lago sono diventate zone protette.} Diceva: ``Vede, è che mi rilassa tanto; non
vengo qui tanto per pescare, ma perché stare qui mi rilassa''. Andava a
pesca tutte le domeniche solo per rilassarsi. A me pareva in buona
salute, perfino un po' rotondetta, non faceva certo la fame. Non aveva
bisogno di pescare per vivere. Le risposi: ``Be', allora potrebbe - dato
che non ha bisogno di pescare per vivere, ha abbastanza soldi, spero,
per comprarsi il pesce - quando compreremo il lago potrebbe venire qui a
meditare. Non è necessario pescare''. Ma lei non voleva meditare! Passo a
lamentarsi dei conigli che le mangiavano i cavoli, per cui aveva dovuto
ricorrere a ogni sorta di trappole mortali per dissuaderli. Questa donna
non ha mai riflettuto su nulla. Lesina i cavoli ai conigli, ma lei può
benissimo comprarseli al mercato. I conigli no. I conigli devono
arrangiarsi a mangiare i cavoli di qualcun altro. Però lei non si è mai
aperta a considerare la realtà delle cose, o ciò che è veramente gentile
e benevolo. Non direi che è una persona crudele o insensibile, solo una
borghese ignorante che non ha mai riflettuto sulla natura o compreso il
modo di essere del \textit{Dhamma}. Quindi lei crede che i cavoli siano lì per
lei e non per i conigli, e che i pesci esistano solo perché lei possa
trascorrere una domenica tranquilla divertendosi a torturarli.

Ora, questa capacità di riflettere e osservare è esattamente ciò che
intende il \textit{Buddha} quando parla di liberarsi dalla cieca dipendenza dalle
abitudini e dalle convenzioni. E' una via per liberare il nostro essere
dall'illusione della condizione sensoriale, attraverso la saggia
riflessione sulle cose così come sono. Cominciamo a osservare noi
stessi, il desiderio o l'avversione, l'opacità o l'ottusità della mente.
Non facciamo preferenze, non cerchiamo di creare le condizioni ottimali
per il nostro piacere personale, ma siamo disposti a tollerare anche le
situazioni più spiacevoli o dolorose allo scopo di comprenderle per
quelle che sono, ed essere in grado di lasciar andare. Cominciamo a
liberarci dalla tendenza a fuggire da ciò che non ci piace. Cominciamo
anche a essere molto più attenti a come viviamo. Quando si vedono i
fatti con chiarezza, viene proprio voglia di essere molto, molto
scrupolosi in ciò che si fa e si dice. Non si ha nessuna intenzione di
vivere alle spese di altre creature. Non si pensa che la propria vita
sia molto più importante della vita di chiunque altro. Si comincia a
percepire la libertà e la leggerezza del vivere in armonia con la
natura, al posto della pesantezza dello sfruttare la natura per il
proprio tornaconto. Quando aprite la mente alla verità, vi accorgete che
non c'è nulla da temere. Ciò che sorge passa, ciò che nasce muore e non
è il sé; quindi la sensazione di essere prigionieri dell'identificazione
con questo corpo umano sfuma. Non ci vediamo come entità isolate e
alienate sperdute in un universo misterioso e inquietante. Non ci
sentiamo più sopraffatti, alla ricerca di un angolino a cui aggrapparci
per sentirci sicuri, perché siamo in pace con l'universo. Siamo tutt'uno
con la Verità.

