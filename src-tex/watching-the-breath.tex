
L'ānāpānasati\footnote{Ānāpānasati: letteralmente consapevolezza – sati –
dell'ispirazione ed espirazione} è un metodo in cui si concentra la mente sul
respiro; perciò, che siate già esperti o abbiate deciso che è una
partita persa, c'è sempre un momento buono per fare attenzione al
respiro. E' un'occasione per coltivare il samādhi, la concentrazione,
chiamando a raccolta tutta la vostra attenzione sulla sensazione di
respirare. Quindi in questo momento concentratevi con il massimo impegno
su quest'unica cosa per la durata di un'inspirazione e la durata di
un'espirazione. Non cercate di farlo per quindici minuti, perché non ci
riuscireste mai se fosse questo il lasso di tempo stabilito per la
concentrazione su un oggetto. Per tale ragione riferitevi solo alla
durata di un'inspirazione e di un'espirazione.
Riuscirci è questione più di pazienza che di forza di volontà, perché
l'attenzione tende a divagare e occorre sempre riportarci pazientemente
al respiro. Se ci accorgiamo di esserci distratti, notiamo di che si
tratta: può essere perché tendiamo a essere troppo energici all'inizio
senza poi sostenere lo sforzo, ci sforziamo troppo senza sostenere
l'energia. Quindi prendiamo la durata di un'inspirazione e la durata di
un'espirazione come lasso di tempo in cui applicare lo sforzo di
sostenere l'attenzione. Fate uno sforzo al principio dell'espirazione e
sostenetelo fino alla fine, e ricominciate nello stesso modo con
l'inspirazione. Alla fine verrà naturale, e quando il tutto sembra
accadere senza sforzo si dice di aver raggiunto il samādhi.

All'inizio sembra faticoso o perfino impossibile, perché non ci siamo
abituati. La mente per lo più è abituata al pensiero associativo. E'
allenata dalla lettura dei libri o simili a procedere parola per parola,
a formulare pensieri e concetti basati sulla logica e sulla ragione.
Invece ānāpānasati è un tirocinio di tipo diverso, in cui l'oggetto su
cui ci si concentra è così semplice da non rivestire il minimo interesse
sul piano intellettuale. Quindi non si tratta di provare interesse, ma
di produrre uno sforzo e usare questa funzione naturale del corpo come
oggetto di concentrazione. Il corpo respira, che ne siamo consapevoli o
no. Non è come il pranayama, in cui il respiro ci serve per sviluppare
certe facoltà; si tratta piuttosto di sviluppare la concentrazione e la
presenza mentale attraverso l'osservazione del respiro, il respiro
normale così com'è in questo momento. Come in tutte le cose, anche qui
c'è bisogno di esercizio per riuscire; in teoria è tutto chiarissimo, ma
nella pratica quotidiana ci si può facilmente scoraggiare.

Quello scoraggiamento, però, che nasce dall'incapacità di ottenere il
risultato voluto, è qualcosa di cui essere consapevoli, perché è proprio
questo che ostacola la pratica. Notate quella sensazione, riconoscetela,
poi lasciatela andare. Tornate di nuovo al respiro. Siate consapevoli
dell'attimo in cui cominciate a sentirvi stufi, irritati o impazienti,
prendetene atto, poi lasciate andare e tornate al respiro.

