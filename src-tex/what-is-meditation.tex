% What is Meditation?

Il termine "meditazione" è ampiamente in uso al giorno d'oggi a
designare una vasta gamma di pratiche. Nel Buddhismo indica due tipi di
meditazione, che si definiscono rispettivamente \textit{samatha} e \textit{vipassanā}.
\textit{Samatha} consiste nel concentrarsi su un oggetto determinato invece di
lasciare la mente a briglia sciolta. Si sceglie un oggetto, ad esempio
la sensazione del respiro, e si rivolge tutta l'attenzione alle
sensazioni prodotte dall'inspirazione e dall'espirazione. Questa pratica
conduce all'esperienza della calma mentale - una tranquillità che è
dovuta all'esclusione di tutti gli altri stimoli che giungono attraverso
i sensi.

Per sviluppare la calma mentale ci si serve (inutile dirlo!) di oggetti
“calmanti”. Se volete una mente eccitata l'ideale è una situazione
eccitante - magari una discoteca - non certo un monastero buddhista! E'
facile concentrarsi sull'eccitazione, vero? E' una vibrazione così forte
che vi risucchia completamente. Al cinema, se il film è veramente
emozionante, si resta come ipnotizzati. Non si deve fare un grande
sforzo per guardare una scena molto eccitante, o romantica, o
avventurosa. Ma per chi non è abituato, osservare un oggetto calmante
può risultare terribilmente noioso. Cosa c'è di più noioso che osservare
il respiro, per chi è abituato a cose più stimolanti? Questo particolare
talento richiede un certo sforzo mentale, perché il respiro non è
interessante, non è romantico, non è avventuroso, non è divertente: è
semplicemente così com'è. Quindi c'è bisogno di sforzo, perché manca lo
stimolo esterno.

In questo tipo di meditazione non si cerca di visualizzare un'immagine;
ci si concentra semplicemente sulla normale condizione del corpo così
com'è in questo momento: si mantiene un'attenzione sostenuta sul
respiro. così facendo, il respiro diventa sempre più sottile e a poco a
poco ci si calma... conosco persone a cui è stata prescritta la pratica
di \textit{samatha} come cura per la pressione alta, perché calma il cuore.

Quindi questa è la pratica della tranquillità. Si possono scegliere
oggetti diversi su cui concentrarsi, esercitandosi a sostenere
l'attenzione fino ad immergersi nell'oggetto, a fondersi con l'oggetto.
C'è la sensazione di essere tutt'uno con l'oggetto su cui ci si
concentra, ed è ciò che si definisce “assorbimento meditativo”.

L'altra pratica è definita \textit{vipassanā}, meditazione di “visione profonda”.
Con la \textit{vipassanā}, il campo dell'attenzione si apre ad abbracciare tutto.
Non si sceglie un oggetto particolare su cui concentrarsi o nel quale
assorbirsi, ma si osserva per comprendere la natura delle cose. Ora, ciò
che possiamo vedere circa la natura delle cose è che l'esperienza
sensoriale nel suo complesso è impermanente. Tutto ciò che si vede, si
ode, si annusa, si gusta e si tocca, tutte le condizioni mentali –
sentimenti, ricordi e pensieri - sono mutevoli condizioni della mente,
che sorgono e passano. Nella \textit{vipassanā}, ogni esperienza sensoriale
osservabile mentre siamo seduti qui è vista attraverso questa
caratteristica dell'impermanenza (o del cambiamento).

Non si tratta di un atteggiamento filosofico o di aderire a una certa
teoria buddhista: l'impermanenza va conosciuta intuitivamente aprendo la
mente all'osservazione, ed essendo consapevoli delle cose così come
sono. Non si tratta di analizzarle partendo dal presupposto che debbano
essere in un certo modo, e se poi non è così cercare di capire perché
non sono come pensiamo che dovrebbero essere. Nella pratica della
visione profonda non cerchiamo di analizzarci, e nemmeno di cambiare le
cose secondo i nostri desideri. In questa pratica ci limitiamo a notare
pazientemente che tutto ciò che sorge passa, mentale o fisico che sia.

Quindi anche gli stessi organi di senso, i rispettivi oggetti e la
coscienza che scaturisce dal contatto fra i due. Poi ci sono le
condizioni mentali di attrazione o repulsione rispetto a quanto vediamo,
annusiamo, gustiamo, sentiamo o tocchiamo; i nomi che gli diamo e le
idee, le parole e i concetti che creiamo attorno all'esperienza
sensoriale. Gran parte della nostra vita si basa su presupposti
infondati che derivano dal non capire e non investigare a fondo la
realtà delle cose. Quindi per chi non è sveglio e consapevole la vita
tende a diventare deprimente o sconcertante, specialmente in occasione
di delusioni o eventi drammatici. In quei momenti ci sentiamo
sopraffatti, perché non abbiamo osservato la natura delle cose.

In termini buddhisti si parla di \textit{Dhamma}, o \textit{Dharma}, che significa appunto
“la realtà delle cose”, “la legge di natura”. Quando osserviamo e
“pratichiamo il \textit{Dhamma}”, apriamo la nostra mente alla realtà delle cose.
Così facendo, non stiamo più reagendo ciecamente all'esperienza
sensoriale ma la comprendiamo, e attraverso la comprensione cominciamo a
lasciarla andare.

Cominciamo a liberarci dal nostro essere sopraffatti o accecati e illusi
dall'apparenza delle cose. Essere consapevoli e svegli non significa
diventarlo, ma appunto esserlo. Quindi osserviamo la realtà delle cose
in questo preciso momento, piuttosto che adoperarci ora per diventare
consapevoli in futuro. Osserviamo il corpo così com'è, seduto qui. Il
corpo appartiene in tutto e per tutto alla natura, vi pare? Il corpo
umano appartiene alla terra, si sostenta grazie a ciò che viene prodotto
dalla terra. Non potete vivere di aria o importare il cibo da Marte e da
Venere. Dovete nutrirvi di ciò che vive e cresce su questa Terra. Quando
il corpo muore torna alla terra, marcisce e si decompone e torna a
essere tutt'uno con la terra. Segue la legge di natura, di creazione e
distruzione, di nascita e morte. Tutto ciò che nasce non resta sempre
nello stesso stato, ma cresce, invecchia e poi muore. Tutto in natura,
incluso l'universo stesso, ha una sua durata, una nascita e una morte,
un principio e una fine. Tutto ciò che percepiamo e che possiamo
immaginare è mutevole, è impermanente. Quindi non potrà mai darvi
soddisfazione duratura.

Nella pratica del \textit{Dhamma}, osserviamo anche questo carattere
insoddisfacente dell'esperienza sensoriale. Nella vostra vita quotidiana
avrete notato che quando vi aspettate di ricavare soddisfazione dagli
oggetti o dalle esperienze sensoriali, potete sentirvi soddisfatti solo
temporaneamente, forse gratificati, momentaneamente felici, ma poi le
cose cambiano. Questo perché non c'è nulla nella coscienza sensoriale
che abbia una qualità o un'essenza permanente. Quindi l'esperienza dei
sensi è sempre mutevole, e noi per ignoranza o per mancanza di
comprensione tendiamo a nutrire grosse aspettative al riguardo. Abbiamo
la tendenza a pretendere, sperare e immaginare ogni sorta di cose, solo
per ritrovarci delusi, disperati, addolorati e spaventati. E sono
proprio queste nostre aspettative e speranze a condurci alla
disperazione, all'angoscia, alla tristezza e al dolore, all'afflizione,
alla vecchiaia, alla malattia e alla morte.

Si tratta quindi di un modo per esplorare la coscienza sensoriale. La
mente è capace di astrazione, può generare ogni sorta di idee e di
immagini, pensare cose sublimi o estremamente volgari. C'è tutta una
gamma di possibilità, dagli stati mentali più sublimi di beatitudine e
di estasi alle miserie del dolore più crudo: dal paradiso all'inferno,
per usare termini più coloriti. Ma non c'è un paradiso permanente o un
inferno permanente, né d'altronde alcuno stato permanente che possa
essere percepito o immaginato. Nella meditazione, quando si comincia a
prendere coscienza dei limiti, dell'insoddisfazione e del cambiamento
connaturati a ogni esperienza sensoriale, si comincia anche a percepire
che tutto questo non sono io e non è mio, ma \textit{anattā}, non-sé.

Quindi, prendendone coscienza, cominciamo ad affrancarci
dall'identificazione con le condizioni sensoriali. E questo avviene non
sull'onda dell'avversione, ma comprendendole per quelle che sono. E' una
verità che va realizzata, non un credo. \textit{Anattā} non è un credo buddhista,
è un'esperienza concreta. Ma se non dedicate del tempo al tentativo di
investigarla e di comprenderla, è probabile che tutta la vostra vita
andrà spesa nella convinzione di essere questo corpo. Anche se di quando
in quando potrà capitarvi di pensare "non sono il corpo" - magari
ispirati da una poesia o da un nuovo approcciò filosofico. Potrà
sembrarvi una buona idea, questa di non essere il corpo, ma non lo
avrete sperimentato. Qualche intellettuale potrà affermare che non siamo
il nostro corpo, che il corpo non è il sé; facile a dirsi, ma saperlo è
tutt'altra cosa. Grazie alla pratica della meditazione, attraverso
l'investigazione e la comprensione della realtà delle cose, cominciamo
ad affrancarci dall'attaccamento. Quando aspettative e pretese vengono
meno, è naturale non provare più la disperazione, la tristezza e il
dolore che ne conseguono se non si ottiene ciò che si desidera. Quindi
la meta è questa, il \textit{Nibbāna}\footnote{\textit{Nibbāna}: la Pace attraverso il non-attaccamento, noto
anche come Nirvana.}, la condizione in cui non ci si
aggrappa a nessun fenomeno che abbia un principio e una fine. Quando
lasciamo andare questo insidioso e abituale attaccamento a ciò che nasce
e muore, cominciamo a realizzare l'Immortale.

Alcuni di noi vivono la loro vita semplicemente reagendo alla vita
perché sono condizionati a farlo, come i cani di Pavlov. Se non vi
risvegliate alla realtà delle cose, di fatto non siete altro che
creature intelligenti condizionate, piuttosto che stupidi cani
condizionati. Si può guardare con sussiego ai cani di Pavlov che sbavano
quando suona il campanello; ma notate quanto anche noi ci comportiamo in
modo simile. Questo perché l'esperienza sensoriale è tutta fatta di
condizionamento, non è una persona, un'anima, una sostanza personale.
Questo corpo, le sensazioni, i ricordi e i pensieri sono percezioni
mentali condizionate dal dolore, dall'essere nati come esseri umani,
dall'essere nati in una certa famiglia, dall'appartenenza a una certa
classe, razza e nazionalità; dipendono dall'avere un corpo femminile o
maschile, attraente o non attraente e così via. Tutte queste sono
semplicemente le condizioni che non ci appartengono, che non sono “io”
né “mie”. E queste condizioni seguono le leggi della natura, le leggi
naturali. Non si può dire "non voglio che il mio corpo invecchi"; o
meglio, possiamo dirlo, ma per quanto insistiamo il corpo invecchia lo
stesso. Non possiamo aspettarci che non provi mai dolore, o non si
ammali mai o conservi sempre una vista e un udito perfetti. Ce lo
auguriamo però, non e vero? "Spero di essere sempre in buona salute, di
non diventare mai invalido e avere sempre una buona vista, di non
diventare cieco e conservare un buon udito, così non sarò mai uno di
quei vecchietti a cui bisogna strillare nelle orecchie; di non diventare
un rimbambito e conservarmi padrone delle mie facoltà finché arriverò a
novantacinque anni sveglio, lucido e arzillo e morirò nel sonno senza
dolore". Ci piacerebbe che andasse così. Alcuni di noi camperanno a
lungo e avranno una morte idilliaca, ma domani potremmo perdere
all'improvviso tutti e due gli occhi. E' improbabile, ma potrebbe
succedere! Certo è che il peso della vita si allevia considerevolmente
quando riflettiamo sui suoi limiti intrinseci. Allora sappiamo cosa è
possibile ottenere, cosa possiamo imparare dalla vita. Tanta parte della
miseria umana nasce da aspettative esagerate e dall'impossibilità di
ottenere tutto ciò che avevamo sperato.

Quindi nella meditazione e nella comprensione intuitiva della natura
delle cose vediamo che la bellezza, il sublime, il piacere, sono
condizioni impermanenti tanto quanto il dolore, la miseria e la
bruttezza. Se arrivate a capire questo, siete in grado di godere e
tollerare tutto quanto potrà capitarvi. Di fatto, la lezione della vita
consiste in gran parte nell'imparare a tollerare quello che non ci
piace, in noi stessi e nel mondo che ci circonda; imparare a essere
gentili e pazienti senza fare una tragedia per le imperfezioni
dell'esperienza sensoriale. Possiamo adattarci, tollerare e accettare le
caratteristiche mutevoli del ciclo della nascita e della morte
sensoriali mollando la presa e smettendo di attaccarci. Quando ci
liberiamo dall'identificazione con questo ciclo, sperimentiamo la nostra
vera natura, che è luminosa, limpida, consapevole; ma non è più un fatto
personale, non è “me” o “mio”, niente da conquistare o a cui attaccarsi.
Possiamo attaccarci solo a ciò che non siamo!

Gli insegnamenti del \textit{Buddha} sono solo utili strumenti, modi di osservare
l'esperienza sensoriale che ci aiutano a capire. Non sono comandamenti,
non sono dogmi religiosi da accettare o in cui credere. Sono solo
indicazioni che additano la realtà delle cose. Quindi non ci serviamo
degli insegnamenti del \textit{Buddha} come qualcosa di fine a se stesso a cui
aggrapparci, ma solo per ricordarci di essere svegli, vigili e
consapevoli che tutto ciò che sorge passa.

E' un'osservazione, una riflessione continua e costante sul mondo
sensoriale, perché il mondo sensoriale esercita un'influenza
estremamente potente. Avere un corpo come questo nella società in cui
viviamo ci espone tutti a pressioni incredibili. Tutto si muove così
rapidamente - la televisione e la tecnologia moderna, le macchine –
tutto tende a muoversi a un ritmo vertiginoso. E' tutto così attraente,
eccitante e interessante, ed esercita un forte richiamo sui nostri
sensi. Passeggiando per Londra, notate come i cartelloni pubblicitari
richiamano la vostra attenzione su bottiglie di whisky e sigarette!
L'attenzione viene risucchiata da oggetti che si possono comprare,
sospinta immancabilmente verso una rinascita nell'esperienza sensoriale.
La società materialistica cerca di stimolare l'avidità per farvi
spendere denaro, senza però farvi sentire mai appagati di ciò che avete.
C'è sempre qualcosa di meglio, di più nuovo, di più squisito di quello
che ieri era il più squisito... e così via all'infinito, alienati e
risucchiati negli oggetti dei sensi.

Ma quando entriamo nella sala di meditazione, non siamo qui per
guardarci a vicenda o farci attrarre e risucchiare da questo o
quell'oggetto nella stanza, ma tutto serve a ricordarci di noi stessi.
Ci ricorda di concentrarci su un oggetto tranquillo, o di aprire la
mente e investigare e contemplare la natura delle cose. E' un'esperienza
che va vissuta, da ognuno in prima persona. L'illuminazione degli altri
non ci farà diventare illuminati. Dunque si tratta di un movimento verso
l'interno, non di cercare un illuminato fuori di noi che ci illumini.
Offriamo questa opportunità come incoraggiamento e guida, per dare a chi
è interessato la possibilità di farlo. Qui, generalmente, si può stare
tranquilli che nessuno cercherà di rubarvi la borsetta! Di questi tempi
non si può essere certi di nulla, ma si corrono meno rischi qui che in
Piccadilly Circus; i monasteri buddhisti sono rifugi che facilitano
questa apertura mentale. E' un'opportunità che ci è data in quanto
esseri umani.

In quanto esseri umani, abbiamo una mente capace di riflessione e
osservazione. Potete osservare se siete felici o depressi. Potete
osservare la rabbia, la gelosia o la confusione nella vostra mente.
Quando nel corso della seduta vi sentite confusi e irritati, c'è
qualcosa in voi che lo sa. Potete scegliere se odiare quei sentimenti e
reagire ciecamente, o essere più pazienti e osservare che si tratta di
una condizione temporanea e mutevole di confusione, di rabbia o di
avidità. Un animale invece non può farlo; quando è arrabbiato non c'è
altro, si perde dentro completamente. Provate a dire a un gatto
arrabbiato di osservare la sua rabbia! Con la nostra gatta non c'è stato
verso, lei non è in grado di contemplare l'avidità. Ma io sì, e sono
certo che lo siete anche voi. Se vedo di fronte a me un piatto saporito,
il moto della mia mente non è diverso da quello della gatta Doris. Noi
però possiamo osservare l'attrazione animale verso ciò che ha un buon
odore e un bell'aspetto.

Ciò significa usare la saggezza per osservare l'impulso e comprenderlo.
Ciò che osserva l'avidità non è avidità: l'avidità non può osservare se
stessa, ma ciò che non è avidità può osservarla. Questo osservare è ciò
che chiamiamo “\textit{Buddha}”, la saggezza di \textit{Buddha}, la consapevolezza delle
cose così come sono.

